\chapter{The Contingency of Human Existence}

We are contingent beings. This is the fundamental ontology of the human existence, the metaphysics of our humanity. A human being is no more than a being-for-itself, cloaked in the flimsy body of its own facticity. Our being is defined by our \emph{ontological act}, the constant fleeing of our for-itself away from the in-itself of facticity. But why contingency? Why is our being defined by this nothingness, which lies within us like a worm in the heart? What does it \emph{mean}, in terms of the most concrete and vivid modalities of interaction, that characterise the human experience? This contingency of being is not simply an obscure ontological derivation, like some law of quantum metaphysics only apparent at the smallest scales or the highest abstractions. But rather, nothingness is a fundamental force, which warps the very space-time of the human-reality. The experience of our being is characterised by a mortal struggle of our for-itself, and it is through this conflict in which the most vivid experiences of our human-reality emerge. 

Contingency is at the heart of our being. And the first, and perhaps most personal experience of our for-itself with contingency, manifests through the phenomena of \emph{anguish}: our most immediate conception of the possibility of our non-being.

\section{Anguish as an Existential Anxiety}

What is anguish, as an experience of human-reality? What distinguishes it from other emotions, and why is it \emph{existential}?
Sartre is clear to distinguish anguish from the more ordinary emotion of fear. \textcquote[66]{Sartre}{Anguish is distinguished from fear by the fact that fear is fear of beings in the world, and anguish is anguish before myself.} It is this reflective nature, of an \enquote{anguish before myself,} that holds the key to the ontological meaning of this emotion.
\textcquote[66]{Sartre}{The preparation of artillery that precedes an attack may provoke fear in the soldier undergoing the bombardment, but he will begin to feel anguish when he tries to foresee the behaviour through which he will resist the bombardment, when he wonders if he will be able to \enquote{hold out.}} 
Will I hold out? Or will I crumple, under the stress of my situation?
That question, that uncertainty -- is a hesitation which points towards our being. For I am not steel nor iron, a material object.
The magnitude of my strength is not a rating, a measurement of kilopascals or newton-meters. If I was a being-in-itself, I would simply perform to the adequation of my being, with no questions involved. An iron bar has a tensile modulus, a compressive strength. It will always \emph{be} those properties, never more, but also never any less.

In contrast, as the being-for-itself, I always have the possibility of \emph{not} being, my being. The soldier undergoing bombardment might be trained to withstand fire, to have undergone extensive drills and exercises. Indeed, the soldier should have all the means of resisting the attack, as fortitude is characteristic of a soldier-being. But yet, at the moment of the bombardment, we realise that our resistance is just as equally \emph{possible} as our capitulation. There is no \emph{foundation} for our being, no matter how much we wish it to be. The soldier possesses every opportunity of abandoning his or her post, just as much as they possess the possibility of remaining steadfast upon their ground. \textcquote[69]{Sartre}{In other words, by constituting a specific course of action as \emph{possible}, and precisely because it is \emph{my} possible, I realise that \emph{nothing} can oblige me to take this action.} Nothing can oblige me to be who I am -- no argument can be made from determinism or \enquote{psychophysiological parallelism.} I can be a coward, just as much as I can be brave. In fact, the very meaning of bravery must come from the \emph{possibility} of cravenness. Hence, this is why anguish often comes about in moments of danger, daring, or difficulty -- for in these trials of character, we are forced to strip away the more mundane material which hides our ontology -- and we come face-to-face with the void within us, the dizzying vertigo of not-being, our being.

% \section{Contingency as a Source of Freedom}
Thus, as human beings -- we are cursed with this experience of anguish. This anguish is inescapable, for \emph{freedom} comes from the same contingency that yields our angst. Indeed, the ability to derive freedom as the non-being of our being is one of Sartre's ontological innovations -- it is the reason why nothingness is essential to our ontology. Anguish is one of the ways in which we can directly experience our freedom, as an immediate conception of our being (as opposed to a more abstract, rational conception of it). \textcquote[66]{Sartre}{It is in anguish that man becomes conscious of his freedom, or alternatively, anguish is freedom's mode of being as consciousness of being; it is in anguish that freedom is, in its being, in question for itself.} This metaphysical movement is sophisticated, for it allows us to ground our freedom on an ontological basis, making it inseparable from our humanity. \textcquote[579]{Sartre}{In this way freedom is not \emph{a} being: it is man's being, i.e. his nothingness of being.} But this freedom comes at a cost, which is that deep, ontological insecurity. \textcquote[579]{Sartre}{Human-reality is entirely abandoned, without help of any kind, to the unbearable necessity of making itself be, right down to the last detail.} This abandonment is an ontological neurosis, a fundamental anxiety of our self. And as Sartre goes on to demonstrate -- the most visceral and ontologically threatening manifestation of our existential anxiety, comes from our \emph{being-for-the-Other}.


\section{The Existential Terror}

