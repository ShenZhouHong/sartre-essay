\chapter{The Contingency of Human Existence}

We are contingent beings. This is the fundamental ontology of the human existence, the metaphysics of our humanity. A human being is no more than a being-for-itself, cloaked in the flimsy body of its own facticity. Our being is defined by our \emph{ontological act}, the constant fleeing of our for-itself away from the in-itself of facticity. But why contingency? Why is our being defined by this nothingness, which lies within us like a worm in the heart? What does it \emph{mean}, in terms of love or relationships -- which are the most concrete modalities of interaction which characterise the human experience? This contingency of being is not simply an obscure ontological derivation, like some law of quantum metaphysics only apparent at the smallest scales or the highest abstractions. But rather, nothingness is a fundamental force, which warps the very space-time of the human-reality. The experience of our being is characterised by a mortal struggle of our for-itself, and it is through this conflict in which the most vivid experiences of our human-reality emerge. 

Contingency is at the heart of our being. And the first, and perhaps most personal experience of our for-itself with contingency, manifests through the phenomena of \emph{anguish}: our most immediate conception of the possibility of our non-being.

\section{Anguish as an Existential Anxiety}

What is anguish, as an experience of human-reality? What distinguishes it from other emotions, and why is it \emph{existential}?
Sartre is clear to distinguish anguish from the more ordinary emotion of fear. \textcquote[66]{Sartre}{Anguish is distinguished from fear by the fact that fear is fear of beings in the world, and anguish is anguish before myself.} It is this reflective nature, of an \enquote{anguish before myself,} that holds the key to the ontological meaning of this emotion.
\textcquote[66]{Sartre}{The preparation of artillery that precedes an attack may provoke fear in the soldier undergoing the bombardment, but he will begin to feel anguish when he tries to foresee the behaviour through which he will resist the bombardment, when he wonders if he will be able to \enquote{hold out.}} 
Will I hold out? Or will I crumple, under the stress of my situation?
That question, that uncertainty -- is a hesitation which points towards our being. For I am not steel nor iron, a material object.
The magnitude of my strength is not a rating, a measurement of kilopascals or newton-meters. If I was a being-in-itself, I would simply perform to the adequation of my being, with no questions involved. An iron bar has a tensile modulus, a compressive strength. It will always \emph{be} those properties, never more, but also never any less.

In contrast, as the being-for-itself, I always have the possibility of \emph{not} being, my being. The soldier undergoing bombardment might be trained to withstand fire, to have undergone extensive drills and exercises. Indeed, the soldier should have all the means of resisting the attack, as fortitude is characteristic of a soldier-being. But yet, at the moment of the bombardment, we realise that our resistance is just as equally \emph{possible} as our capitulation. There is no \emph{foundation} for our being, no matter how much we wish it to be. The soldier possesses every opportunity of abandoning his or her post, just as much as they possess the possibility of remaining steadfast upon their ground. \textcquote[69]{Sartre}{In other words, by constituting a specific course of action as \emph{possible}, and precisely because it is \emph{my} possible, I realise that \emph{nothing} can oblige me to take this action.} Nothing can oblige me to be who I am -- no argument can be made from determinism or \enquote{psychophysiological parallelism.} I can be a coward, just as much as I can be brave. In fact, the very meaning of bravery must come from the \emph{possibility} of cravenness. Hence, this is why anguish often comes about in moments of danger, daring, or difficulty -- for in these trials of character, we are forced to strip away the more mundane material which hides our ontology -- and we come face-to-face with the void within us, the dizzying vertigo of not-being, our being.

% \section{Contingency as a Source of Freedom}
Thus, as human beings -- we are cursed with this experience of anguish. This anguish is inescapable, for \emph{freedom} comes from the same contingency that yields our angst. Indeed, the ability to derive freedom as the non-being of our being is one of Sartre's ontological innovations -- it is the reason why nothingness is essential to our ontology. Anguish is one of the ways in which we can directly experience our freedom, as an immediate conception of our being (as opposed to a more abstract, rational conception of it). \textcquote[66]{Sartre}{It is in anguish that man becomes conscious of his freedom, or alternatively, anguish is freedom's mode of being as consciousness of being; it is in anguish that freedom is, in its being, in question for itself.} This metaphysical movement is sophisticated, for it allows us to ground our freedom on an ontological basis, making it inseparable from our humanity. \textcquote[579]{Sartre}{In this way freedom is not \emph{a} being: it is man's being, i.e. his nothingness of being.} But this freedom comes at a cost, which is that deep, ontological insecurity. \textcquote[579]{Sartre}{Human-reality is entirely abandoned, without help of any kind, to the unbearable necessity of making itself be, right down to the last detail.} This abandonment is an ontological neurosis, a fundamental anxiety of our self. And as Sartre goes on to demonstrate -- the most visceral and ontologically threatening manifestation of our existential anxiety, comes from our \emph{being-for-the-Other}.


\section{The Existential Terror}

What is the Other? As a being-for-itself, in the primordial past of the Sartrean Cosmogony, we may very well begin with no conception of the Other. To understand the \emph{existential terror}, let us first take a step back into that world. 
% Edit this at some point.
Imagine being at the dawn of the baryosynthesis. I am a for-itself, a first-class transcendence who is the source of my own nothingness. I live in a world of oak trees and bubbling brooks -- all of which act according to principles \enquote{not of my own will.} What sort of existence is this? What manner of being is available, to this primordial for-itself? It is a world of phenomena, of course -- for we have already achieved the asymmetry of the for-itself and the in-itself. It is also a free world, founded on the same nothingness that yielded phenomena in the first place. I am free to explore this world through the projects of my freedom -- to grasp the in-itself of nature, as instruments of my will. In this way, it is a world of \emph{meaning}. My being grasps the situation of its own facticity in the light of its aims and desires, and in doing so -- I endow the world with relations, all of which leads back towards me. That bubbling brook is a plenitude of being, insofar it is an in-itself. But it can \emph{mean} fresh water, a respite from heat, or a treacherous obstacle, depending on whatever aims I project. There is still anguish in this world, yes. But whatever questions I have of my being -- it is still fundamentally my being, my own \emph{ontological act}. This is the Edenic existence of the primordial for-itself, a Ptolemaic ontology.

This is the world which we fall from, the moment, we meet the \emph{Other}. For what \emph{is} the Other? The Other is not a being-in-itself. It is a being-for-itself, a first-class transcendence just like me. The moment our eyes meet, I recognise myself in the Other -- and this recognition yields an indisputable proof of his Other-ness as a transcendence. \textcquote[338]{Sartre}{The Other is not an object. He remains, in his connection to me, a human reality; the being through which he determines me in my being is his pure being.} But yet, the Other \emph{is not} myself -- and all of a sudden, my privileged position as the source of all meaning is jeopardised: 

\blockcquote[350 -- 351]{Sartre}{[The Other] appears as a pure disintegration of the relations that I apprehended between the objects in my universe \ldots\ Thus, all of a sudden, an object has appeared that has stolen the world from me. Everything is in place, everything still exists for me, but now an invisible and frozen flight towards a new object penetrates everything. The Other's appearing in the world corresponds, therefore, to a frozen sliding away of the universe in its entirety, to a decentering of the world that undermines the centralisation I simultaneously impose.}

\noindent
In this manner, the existence of the Other degrades the primacy of our own being -- and through this act of \enquote{decentering}, we yield \emph{objectivity}. For the meaning of the world is no longer entirely dependent on me. It is factored by the presence of the Other, who is just as capable of grasping the world as objects of their consciousness. And it is through this objectivity, that we experience \emph{existential terror} at our encounter with the Other. We are forced, to be face-to-face with our contingency, the moment we meet the Other's \emph{gaze.} For when the Other looks at me, I become an \emph{object} of the Other. I am frozen at the moment of the \emph{look}, not as the free transcendence of the for-itself, but as an in-itself, a being-in-the-world. For Sartre, the most immediate and visceral perception of this objectification takes the form of the phenomenon of \emph{shame}:

\blockcquote[357 -- 358]{Sartre}{It is shame or pride that reveals the Other's look to me, and myself at the furthest point of [the Other's] gaze; they make me \emph{live,} and not \emph{know}, the situation of being looked at. Shame is shame of \emph{oneself}, it is the \emph{recognition} that I really \emph{am} this object being looked at and judged by the Other.}

\noindent
Why is shame so indicative of the ontological dimension of this \emph{existential terror?} It is because the motion of being-looked-at (by the Other) is a twofold movement, that highlights our contingency both as an in-itself, but also as a lack of being as the for-itself.

In the moment of the look, we are first objectified by the Other. We are robbed of our transcendence, reduced to an in-itself. This reduction comes from the fact that the Other is free -- \textcquote[359]{Sartre}{the Other's freedom is revealed to me through the disturbing indeterminacy of the being that I am for him.} But it is this indeterminacy of the \enquote{being for him} that puts us into the most acute crisis. For not only am I a being held captive as the object of an alien consciousness, that being which \emph{is me} has a foundation that is completely contingent on the Other's will. What is my being-for-the-Other? It is \emph{my} being --  but what \emph{is} it? That depends on the Other -- and whatever forms it take, good or bad -- is both me, but also fundamentally inaccessible to me. \textcquote[365]{Sartre}{To be looked at is to grasp oneself as the \emph{unknown object of unknowable assessments}, and in particular, evaluative assessments \ldots\ being seen constitutes me as a \emph{defenceless being for a freedom that is not my own}.} Thus, my being is not only contingent on the basis of my own non-being -- but my being is contingent on the basis of a freedom that is both outside of me, and fundamentally inaccessible to me.

Hence, this is why the Other is a source of ontological terror. If anguish is the existential anxiety of not having a foundation for our being, then our being-for-the-Other is the existential terror of being contingent solely on the whims of a foreign freedom. 

\blockcquote[366]{Sartre}{Insofar as I am the object of values that enter in to qualify me -- without my being able either to act upon this qualification or even to know it -- I am enslaved. By the same token, insofar as I am the instrument of possibilities that are not my possibilities, \ldots\ which negate my transcendence in order to constitute me as a means towards ends of which I am ignorant, \emph{I am in danger}. And this danger is not an accident, but the permanent structure of my being-for-the-Other.}

\section{The Possibility of Salvation through the Other}

\noindent
But yet, there is something slightly different about this contingency that we experience at the hands of the Other. With anguish, we experience the possibility of our non-being, through the question of our being. However, when we encounter the Other -- what we experience not necessarily our being as a \emph{contingent being}. Rather, we experience our being \emph{as a being}, that is \emph{contingent at the hands of the Other}. This is the very definition of our \emph{being-for-the-Other}, which Sartre asserts is an integral modality of our being-for-itself \autocite[479]{Sartre}.

The ontology of being contingent in the hands of the Other is the source of our existential terror in the first place. However, it is also the source of an intoxicating possibility.\footnote{Sartre explores this \enquote{intoxicating possibility} through a rather novel treatment on the ontology of \emph{seduction}, in \autocite[492]{Sartre}.} \emph{Could we find a foundation for our being, through the freedom of the Other?} We are endangered by the Other's freedom -- threatened by it. Our being-for-the-Other is contingent on the Other's whims. However, if we can somehow make ourselves \emph{not contingent}, in the Other's freedom -- than we would be able to find a foundation for our own being, \emph{in the Other.}

This is the intoxicating possibility of finding a foundation for our being through the means of the Other. It is the possibility of becoming \textcquote[734]{Sartre}{as a for-itself \emph{a being that is what it is.}} How can we act upon the Other's freedom, so that it does not endanger us, but rather serves as a foundation for our being? What can we do to lose our contingency, through our being-for-the-Other? How can I \enquote{reclaim my being,} which lies within the Other's gaze, \textcquote[483]{Sartre}{like Tantalus's meal, and found it by my particular freedom?} This is the project of our being, in the face of the Other. It is a concrete manifestation of our ontological fear, of our conflict with the contingency within us. It is the \enquote{unrealisable ideal,} in which we seek our foundation through the Other's freedom, the \textcquote[485]{Sartre}{basic relation to the Other \ldots\ through which I aim to realise this value.} What is this project?

\noindent
This project is \emph{love.}