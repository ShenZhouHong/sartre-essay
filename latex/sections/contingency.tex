\chapter{The Contingency of Human Existence}

% Be Goddamned passionate. This is the **KNIFE'S EDGE** of your thesis
% Talk about love.
% Getting out early what's at stake.
% The paradox: without contingency, we would be things. But WITH contingency, love and teleos becomes bleak.

We are contingent beings. This is the fundamental ontology of the human existence, the metaphysics of our humanity. A human being is no more than a being-for-itself, cloaked in the flimsy body of its own facticity. Our being is defined by our \emph{ontological act}, the constant fleeing of our for-itself away from the in-itself of facticity. Why are we contingent? What is the meaning of this contingency? What is at stake here, when our human-existence is defined by our own nothingness, which haunts us in every step we take? Is our contingency is like a curse, a hidden unravelling? Unlike the perfect adequation of the being-in-itself, we can \emph{never} hope to coincide with our being, to truly \emph{be} in-itself. We all contain the poisonous seed of negation, the question of our own being. This uneasiness manifests itself as the phenomena of \emph{anguish}, which Sartre dissects and presents to us, with all the cold certainty of a tumour at the hands of a pathologist. For what is anguish, other than a deep, visceral acknowledgement, of the contingency of our being?

% Talk about anguish as an introduction to the bad part of our contingency

Sartre is clear to distinguish anguish from the more ordinary emotion of fear. \textcquote[66]{Sartre}{Anguish is distinguished from fear by the fact that fear is fear of beings in the world, and anguish is anguish before myself.} It is this reflective nature, of an \enquote{anguish before myself,} that holds the key to the ontological meaning of this emotion.
\textcquote[66]{Sartre}{The preparation of artillery that precedes an attack may provoke fear in the soldier undergoing the bombardment, but he will begin to feel anguish when he tries to foresee the behaviour through which he will resist the bombardment, when he wonders if he will be able to \enquote{hold out.}} 
Will I hold out? Or will I crumple, under the stress of my situation?
That question, that uncertainty -- is a hesitation which points towards our being. For I am not steel nor iron, a material object.
The magnitude of my strength is not a rating, a measurement of kilopascals or newton-meters. If I was a being-in-itself, I would simply perform to the adequation of my being, with no questions involved. An iron bar has a tensile modulus, a compressive strength. It will always \emph{be} those properties, never more, but also never any less.

In contrast, as the being-for-itself, I always have the possibility of \emph{not} being, my being. The soldier undergoing bombardment might be trained to withstand fire, to have undergone extensive drills and exercises. Indeed, the soldier should have all the means of resisting the attack, as fortitude is characteristic of a soldier-being. But yet, at the moment of the bombardment, we realise that our resistance is just as equally \emph{possible} as our capitulation. There is no \emph{foundation} for our being, no matter how much we wish it to be. The soldier possesses every opportunity of abandoning his or her post, just as much as they possess the possibility of remaining steadfast upon their ground. \textcquote[69]{Sartre}{In other words, by constituting a specific course of action as \emph{possible}, and precisely because it is \emph{my} possible, I realise that \emph{nothing} can oblige me to take this action.} Nothing can oblige me to be who I am -- no argument can be made from determinism or \enquote{psychophysiological parallelism.} It is from this very nothingness, this negation of the self -- that we found the source of our own \emph{freedom.} \textcquote[66]{Sartre}{It is in anguish that man becomes conscious of his freedom, or alternatively, anguish is freedom's mode of being as consciousness of being; it is in anguish that freedom is, in its being, in question for itself.}

% Segue into talking about freedom, and how freedom arises from our contingency since we do not have a fixed being.

\section{Contingency as the Source of Freedom}

% Talk about the bad part of freedom, and then transition into talking about concrete relations w/ the Other

\section{Contingency in Relations with the Other}

% In talking about Other, talk about love, and the paradox of it

% Make sure to remember to talk about the following concept: the 'ontological-complexity' of a metaphysics. Where there is roughly a balance to be had between solipsism, and nihilism, but a complex array of options between the two which defines ontological complexity:

\section{Ontological Complexity and Metaphysics}
 
% Solipsism - the ontology is too simple, all is reduced to one
% Finite ontology - the ontology of chess and checkers, a finite universe.
% Deterministic ontology -- the ontological complexity expands at a constant rate
% A Free ontology - the ontology expands non-linearly, but just right -- there is freedom
% Nihilism - the ontology is too general, there is no being at all

% Talk about how we are different from the Greek Gods