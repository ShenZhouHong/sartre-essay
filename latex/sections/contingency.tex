\chapter{The Contingency of Human Existence}

We are contingent beings. This is the fundamental ontology of the human existence, the metaphysics of our humanity. A human being is no more than a being-for-itself, cloaked in the flimsy body of its own facticity. We are defined by our nothingness -- our very being is a \emph{flight}, which constantly flees from the in-itself of our facticity. This contingency is a deep part of us -- and it is neither obscure, nor hidden in its manifestations. Instead, it is a \emph{conflict} -- as beings-for-itself, we are forced to confront and grapple with our contingency, in almost every thing we do. It is our mortal struggle with contingency, that brings about the most vivid experiences of the human condition. 

\section{The Existential Anxiety}

It all begins with \emph{anguish}. Anguish is our consciousness of contingency, it is the most immediate conception of our possibility of non-being. Sartre is clear to distinguish anguish from the more ordinary emotion of fear. \textcquote[66]{Sartre}{Anguish is distinguished from fear by the fact that fear is fear of beings in the world, and anguish is anguish before myself.} It is this reflective nature, of an \enquote{anguish before myself,} that holds the key to the ontological meaning of this emotion. \textcquote[66]{Sartre}{The preparation of artillery that precedes an attack may provoke fear in the soldier undergoing the bombardment, but he will begin to feel anguish when he tries to foresee the behaviour through which he will resist the bombardment, when he wonders if he will be able to \enquote{hold out.}} 
Will I hold out? Or will I crumple, under the stress of my situation?
That question, that uncertainty -- is a hesitation which points towards our being. For I am not steel nor iron, I am no material object.
The magnitude of my strength is not a rating, a measurement of kilopascals or newton-meters. If I was a being-in-itself, I would simply perform to the adequation of my being, with no questions involved. An iron bar has a tensile modulus, a compressive strength. It will always \emph{be} those properties, never more, but also never any less.

In contrast, as the being-for-itself, I always have the possibility of \emph{not} being, my being. The soldier undergoing bombardment might be trained to withstand fire, to have undergone extensive drills and exercises. Indeed, the soldier should have all the means of resisting the attack, as fortitude is characteristic of a soldier-being. But yet, at the moment of the bombardment, we realise that our resistance is just as equally \emph{possible} as our capitulation. There is no \emph{foundation} for our being, no matter how much we wish it to be. The soldier possesses every opportunity of abandoning his or her post, just as much as they possess the possibility of remaining steadfast upon their ground. \textcquote[69]{Sartre}{In other words, by constituting a specific course of action as \emph{possible}, and precisely because it is \emph{my} possible, I realise that \emph{nothing} can oblige me to take this action.} Thus, anguish is the most immediate manifestation of our contingency. It is a deep, inborn anxiety -- an anxiety of our ontology. Hence, it is the \emph{existential anxiety}.

\section{The Existential Terror}

Unfortunately, anguish is only one of the many ways in which we experience our contingency. It may be the most immediate experience, and perhaps in some ways, it may even be the most private. But anguish is decidedly not the only manner in which our contingency manifests -- it is overshadowed by another. If anguish is our existential anxiety, then this is our \emph{existential terror.} We are talking about our contingency in the eyes of the Other -- that begins at the very instance, we are confronted by the Other's gaze. 

What does it mean to be confronted by the Other? What does it mean to meet a being in the world, that is both not a being-in-itself, but also not myself? Why does it lead to existential terror? 
As a for-itself in the world, I am 
