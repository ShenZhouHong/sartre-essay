\chapter{Introduction}

% Why is this inquiry important in the first place?

% Why Being and Nothingness?

% What is the most exciting thing about Sartre's Being and Nothingness?

% This paragraph essentially answers the question of "Why is metaphysics important?"

What is \emph{being?} \textquote{Why are there beings at all, instead of nothing?} That was the question, presented in the words of Martin Heidegger, as the \emph{fundamental question of metaphysics} \autocite[1]{Heidegger}. The question of being is both the broadest question, as well as the deepest -- those answer must account both for \textcquote[4]{Heidegger}{some elephant in a jungle in India, just as much as some chemical oxidation process\footnote{First published in 1935, Martin Heidegger's astronomical quip predated the Viking lander (and any practical investigations into the being of Martian surface chemistry) by more than 41 years.} on the planet Mars.} Our ability to give an adequate account of being is important, not just on the basis of some abstract, theoretical desire, but as a matter of practical utility too -- for to understand the \emph{being} of a human, is to know what is the \textcquote[11]{Aristotle}{characteristic activity} of a \emph{human} being. The broad and fundamental generality that Metaphysics holds in relation to the rest of Philosophy is akin to the relationship between Physics and Mechanics -- to understand the laws of the former would allow us to derive the facts of the latter. This makes ontology akin to the theoretical physics of Philosophy, with being as the constitutive element of its science -- the \emph{fundamental force} of Metaphysics.

% The next paragraph essentially answers the question of "Why is Sartre's Metaphysics a good explanation, as opposed to Kant or anyone else?"

It is this question of being which interests me, as a strong theoretical foundation in ontology can lead to further (and even unexpected) applications in more subsequent branches of philosophy. Jean-Paul Sartre's \emph{Being and Nothingness} is a monograph which presents a complete, and self-sufficient \emph{system} of ontology, that offers a stronger theoretical underpinning than prior systems we have studied. \emph{Being and Nothingness} inherits from a phenomenological background which explicitly aims \textcquote[1]{Sartre}{to eliminate a number of troublesome dualisms from philosophy, and to replace them with the monism of the phenomenon.} This approach is entirely different, when compared with Immanuel Kant's transcendental metaphysics, whose \emph{Critique of Pure Reason} presents and is dependent on a strong and inseparable dichotomy between the noumena and phenomena -- a \enquote*{troubling dualism,} in other words. Sartre rejects this dualism: we postulate that the being of an existent is entirely in the existent's appearances, and within a few short strokes lay out the opening propositions of an entirely novel \emph{phenomenological ontology.}

% Now we present the question of contingency, that comes from Sartre's ontology

% Explain briefly how the question of contingency comes from nothingness-- hence in order for us to explore this question properly, we need to first understand where nothingness comes from 

% Begin an explanation of the genesis of contingency from nothingness, using a physical analogy of Cosmogony.