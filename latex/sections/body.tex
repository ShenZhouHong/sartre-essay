\chapter{The Cosmogony of Sartrean Ontology}

% Explain briefly how the question of contingency comes from nothingness-- hence in order for us to explore this question properly, we need to first understand where nothingness comes from 

% Begin an explanation of the genesis of contingency from nothingness, using a physical analogy of Cosmogony.
% We're basically using the literary device of an extended metaphor with (physical) cosmogony, of an 'ontological Big-Bang' to set up a ordered exposition of Sartre's ontology.

What is contingency? What is the nothingness inside our being, and how is the relationship between our being and nothingness a relationship of contingency? How did contingency arise in the first place? In order to answer these questions, which are more subsequent propositions of Sartre's ontology -- we must first take a look at nothingness, and begin with the postulates and definitions of Sartre's system. We must look at how being emerges in the first place -- and trace our way through the history of genesis in which the ontology of our human-reality emerges. This process is an ontological cosmogony, and the progression of our science shares familiar motifs with its counterpart in physical cosmology: complete with its own epochs and symmetry-breaking. Hence, we will begin our inquiry into the contingency of our being, with the very cosmogony of Sartrean ontology itself.

% This extended metaphor will take us through the first three major parts of Sartre's Being and Nothingness:
%  1. The Problem of Nothingness -- where it comes from
%      1. First, we begin with existents -- and the phenomena which we perceive
%      2. We acknowledge the 'factual neccessity' of there being a consciousness to perceive phenomena
%      3. Next, we present the problem where phenomena must have its own being independent of the being of conscousness
%      4. We present the 'Baryogenesis' of ontology, where how the negative act of consciousness begets objectivity.

Our investigation begins from the foundation of phenomena -- the basic realm of derivative ontological data that is readily accessible to our metaphysics. We happen to live in a world of phenomena -- a rich plenum of perceptions that forms the infinite state-space of human-reality. How can we find being, starting from the raw data of the phenomena? Very quickly, we realise that there is a distinction between the \emph{phenomenon-of-being}, and the \emph{being-of-phenomena} -- at least, a distinction that is possible in the infancy of our incipient ontology. What we seek to grasp is the \emph{being-of-phenomena} -- the universal, ontological basis for all phenomena. In contrast, the phenomenon-of-being is the more superficial \emph{appearance} of any arbitrary being -- much more accessible to us, but not necessarily the same as the \emph{being-of-phenomena}. \textcquote[6]{Sartre}{Is the being that is disclosed to me, that \emph{appears} to me, the same in nature as the being of the existents that appear to me?} Can the former (the phenomena-of-being), be reduced to or otherwise lead us to the latter (the being-of-phenomena)? 

Unfortunately, such a reduction is not possible -- we cannot reduce the subsequent phenomena-of-being to the more fundamental being-of-phenomena. The being of an object cannot come from the object itself: \textcquote[6]{Sartre}{it is not possible, for example, to define being as a \emph{presence}, since \emph{absence} also discloses being, since not being \emph{there} is still a way of being.} This impossibility of reducing the being-of-phenomena to the phenomena-of-being is explored rigorously by Sartre in his introduction, for even more sophisticated attempts at such a reduction is ultimately an appeal to \emph{knowledge} as a foundation for being -- with knowledge necessarily defined as the simple ratio or proportionality between an existent's being and it's appearing, i.e. phenomena \autocite[8]{Sartre}. Hence, the being-of-phenomena is by necessity separate and irreducible from the phenomena-of-being -- which is to say that the being of phenomena lies \emph{outside} of the object of phenomena itself. Hence, Sartre concludes that the being of phenomena is \emph{transphenomenal}: literally outside the phenomena.

%  2. Being for-itself
%      1. We'll need to begin with an examination self-presence, as the 'Baryon asymmetry' of ontology 
%  3. Being for the Other
%      1. We'll have to examine how without the other, the world is merely 'psychological'
% We will save the last part for later

% The Problem of Nothingness:

\chapter{The Contingency of Human Existence}

\chapter{The Teleology of Contingency}

\chapter{Conclusion}
