\chapter{The Cosmogony of Sartrean Ontology}
% Explain briefly how the question of contingency comes from nothingness-- hence in order for us to explore this question properly, we need to first understand where nothingness comes from 

% Begin an explanation of the genesis of contingency from nothingness, using a physical analogy of Cosmogony.
% We're basically using the literary device of an extended metaphor with (physical) cosmogony, of an 'ontological Big-Bang' to set up a ordered exposition of Sartre's ontology.

What is contingency? What is the nothingness inside our being, and how is the relationship between our being and nothingness a relationship of contingency? How did contingency arise in the first place? In order to answer these questions, which are more subsequent propositions of Sartre's ontology -- we must first take a look at nothingness, and begin with the postulates and definitions of Sartre's system. We must look at how being emerges in the first place -- and trace our way through the history of its genesis in which the ontology of our human-reality emerges. This process is an ontological cosmogony, and the progression of our science shares familiar motifs with its counterpart in physical cosmology: complete with its own epochs and symmetry-breaking. Hence, we will begin our inquiry into the contingency of our being, with the very cosmogony of Sartrean ontology itself.

\section{The Foundation of Phenomena}
% This extended metaphor will take us through the first three major parts of Sartre's Being and Nothingness:
%  1. The Problem of Nothingness -- where it comes from
%      1. First, we begin with existents -- and the phenomena which we perceive
%      2. We acknowledge the 'factual neccessity' of there being a consciousness to perceive phenomena
%      3. Next, we present the problem where phenomena must have its own being independent of the being of conscousness
%      4. We present the 'Baryogenesis' of ontology, where how the negative act of consciousness begets objectivity.

Our investigation begins from the foundation of phenomena -- the basic realm of derivative ontological data that is readily accessible to our metaphysics. We happen to live in a world of phenomena -- a rich plenum of perceptions that forms the infinite state-space of human-reality. How can we find being, starting from the raw data of the phenomena? Very quickly, we realise that there is a distinction between the \emph{phenomenon-of-being}, and the \emph{being-of-phenomena} -- at least, a distinction that is possible in the infancy of our incipient ontology. What we seek to grasp is the \emph{being-of-phenomena} -- the universal, ontological basis for all phenomena. In contrast, the phenomenon-of-being is the more superficial \emph{appearance} of any arbitrary being -- much more accessible to us, but not necessarily the same as the \emph{being-of-phenomena}. \textcquote[6]{Sartre}{Is the being that is disclosed to me, that \emph{appears} to me, the same in nature as the being of the existents that appear to me?} Can the former (the phenomena-of-being), be reduced to or otherwise lead us to the latter (the being-of-phenomena)? 

Unfortunately, such a reduction is not possible -- we cannot reduce the subsequent phenomena-of-being to the more fundamental being-of-phenomena. The being of an object cannot come from the object itself: \textcquote[6]{Sartre}{it is not possible, for example, to define being as a \emph{presence}, since \emph{absence} also discloses being, since not being \emph{there} is still a way of being.} This impossibility of reducing the being-of-phenomena to the phenomena-of-being is explored rigorously by Sartre in his introduction, for even more sophisticated attempts at such a reduction is ultimately an appeal to \emph{knowledge} as a foundation for being -- with knowledge necessarily defined as the simple ratio or proportionality between an existent's being and it's appearing, i.e. it's phenomena \autocite[7]{Sartre}. Hence, the being-of-phenomena is by necessity separate and irreducible from the phenomena-of-being -- which is to say that the being of phenomena lies \emph{outside} of the object of phenomena itself. \textcquote[7]{Sartre}{In brief, the phenomenon of being is \enquote{ontological} in the sense in which Saint Anselm's and Descartes's proof is called ontological. It is a call for being: it requires, insofar as it is a phenomenon, a transphenomenal foundation.} Hence, Sartre concludes that the being of phenomena is \emph{transphenomenal}: i.e. outside the phenomena.

If the being-of-phenomena is transphenomenal -- then where can being lie? What external \enquote*{thing}\footnote{Single quotes, i.e. scare quotes, are used to designate concepts provisionally or rhetorically.} can serve as the foundation for the being-of-phenomena? \emph{Another} being is the only choice that is available for us -- for in this stage of the development of our ontological theory, the only two particles of our metaphysics are being and phenomena. Phenomena cannot be the foundation of phenomena -- for that would lead to an infinite regression, a circularity. Hence, only another being can be the basis for phenomena's own being. What is this other being, which serves as the condition for phenomena? Or in other words, if phenomena is merely the appearance of being, but not the being itself -- \emph{then to whom does phenomena appear to?} This being is \emph{consciousness} -- the being \emph{to whom} there are appearances in the first place. And thus, achieve the first breakthrough of our ontological cosmology. In our survey of the background of phenomena, we discover a being-of-phenomena which has no immanent source -- but a transphenomenal origin. This transphenomenal being-of-phenomena points us towards consciousness, like how the cosmic microwave background radiation of the physical universe points us towards the Big Bang. Hence, as with both -- in order to understand the former, we must investigate the latter. At this stage we cannot say anything about the relationship between the being-of-the-phenomena and consciousness -- indeed, we know nothing about the ontological structure of consciousness itself. But now we have a direction for our inquiry, and in a process not-too-dissimilar from an astrophysicist peering back into the progression of the big bang. We trace phenomena towards consciousness, in order to investigate consciousness itself.\footnote{Here I actually depart slightly from the original progression of Sartre's rhetoric. Sartre first explores knowledge in further detail, as knowledge is a more direct embodiment of the relationship between the being-of-phenomena and the phenomena-of-being. It is only after establishing the ontology of knowledge, does Sartre then turn towards the \emph{knower} of the knowledge, which leads us to consciousness. The progression towards knowledge first, then consciousness -- is a necessity of the synthetic nature of presenting a new metaphysics \emph{ab initio}. Thankfully, as his descendents, we may present his ideas analytically.}

\section{The Ontology of Consciousness}

% Talk about the factual neccessity of consciousness

What is the ontology of consciousness? And how does consciousness relate to the being-of-the-phenomena? Does consciousness \enquote*{give} phenomena being, or perhaps \enquote*{generate} being for phenomena? Phenomenon leads us towards consciousness, but what is their exact relationship? Are we certain that there is such a being as consciousness at all? To begin, it does not appear that the being of consciousness is certain -- for while we assert that \enquote{phenomenon must appear to a being}, such an assertion does not seem self-evident. After all, it is possible as a thought experiment for the metaphysician to imagine a world with \enquote*{being}, but without consciousness. But such a world would also fundamentally be without phenomena -- there is \enquote*{being} but no appearance. And further yet, the ontology of this hypothetical world is unstable, for the so-called \enquote*{being} postulated cannot derive its foundation from anywhere. This lemma is presented by Sartre when he states that \textcquote[14]{Sartre}{consciousness is not \emph{possible} before being but instead comprises -- in its being -- the source and condition of all possibility, its existence implies its essence. This is felicitously expressed by Husserl as its [i.e. consciousness's] \enquote{factual necessity.}} This factual necessity is akin to a certain \emph{anthropic principle} of ontology. Consciousness does not necessarily have to exist, but the fact that it exists makes its non-existence inconceivable.

Thus, Sartre asserts the existence of the \emph{being-of-consciousness}. But we have not yet understood the relationship between the being-of-consciousness and the being-of-phenomena yet. What is the relationship between phenomena and consciousness? Is it not the case that phenomena \emph{appears} to consciousness? Can we claim that these appearances are held within consciousness, in the same way that we would speak of \enquote{becoming conscious of \enquote{something?}} Sartre asserts that consciousness is fundamentally \emph{positional}. \textcquote[9]{Sartre}{All consciousness is consciousness of something. In other words, there is no [act of] consciousness that does not \emph{posit} a transcendent object, or if you prefer, consciousness has no \emph{content.}} \textcquote[10]{Sartre}{Consciousness is positional in that it transcends itself to reach an object and is exhausted by this act of positing.} This positional nature of consciousness is also the reason why the ontology of consciousness is a \emph{transcendence}: \textcquote[22]{Sartre}{transcendence is a constitutive structure of consciousness, which is to say that consciousness is born bearing on a being that it is not.} Borrowing from Husserl's phenomenological vocabulary, consciousness in this positional sense is also referred to as \emph{thetic}: pertaining to a \emph{thesis}, an object of consciousness which we have \emph{posited} towards.

% Talk about the twofold nature of consciousness as both positional and pre-cognitive

Now we gain an account of consciousness which ascribes to its ontology the following characteristics: It is contentless, and positional. It posits towards a being that is outside the being-of-consciousness itself, a \emph{transcendence}. Is this account of consciousness ontologically complete, or even sufficient for our purpose? Not yet. For this provisional consciousness of ours, with the above parameters and terms, fails to account for one important (and perhaps even definitive) attribute of consciousness: we are conscious \emph{of} our consciousness. What does it mean to be conscious of our consciousness? \textcquote[10]{Sartre}{The necessary and sufficient condition for a knowing consciousness to be knowledge of its object, is that it should be conscious of \emph{itself} as being this knowledge.} Our thetic, positional consciousness satisfies the first half of this criterion: since our provisional consciousness posits towards its object\footnote{At this stage of our ontology's development, the phrase \emph{object} must contain the bare minimum of ontological significance. When we refer to an \enquote{object of consciousness} in these cases, we strictly mean the grammatical object: the being that which consciousness posits towards. Concepts such as \emph{objectivity} are strictly undefined now, and can only be derived later.}, which we give the name \emph{knowledge}. However, we have failed to account or provision a means for which our consciousness is \textquote{aware of itself as being this knowledge.} To quote Sartre on the strict necessity of this condition:

\blockcquote[10]{Sartre}{
    If my consciousness were not consciousness of being conscious of a table, it would hereby be conscious of the table without being conscious that it was so, or, alternatively, it would be a consciousness that did not know itself, an unconscious consciousness -- which would be absurd.
}

\noindent
Hence, in order to proceed to elucidate the relationship between the being-of-phenomena and the being-of-consciousness, we must first complete our account of the being-of-consciousness with an account for this second term. Like the physicist at the blackboard, we discover that our initial equation does not add up to the sum of the particle which we observe. How can we account for this \enquote*{self-conscious}\footnote{Although the phrase self-consciousness is a more succinct term for the consciousness-of-consciousness, we will refrain from using it -- both in order to stay closer to Sartre's own terminology, as well as to avoid any confusion with \emph{self-presence}, a concept which we will elaborate in later sections.} component of consciousness? Or using Sartre's terminology, this \emph{consciousness-of-consciousness}? The first and theoretically simplest method, is for us to appeal to \emph{reflectivity} as a foundation for this consciousness-of-consciousness. Consciousness is positional. Why not allow consciousness to be conscious of itself? In this case, the thetic object of consciousness would \emph{be} consciousness -- our consciousness posits towards our consciousness, in other words. This approach is simple, but not improperly so -- after all, it possesses a certain theoretical elegance, a balance and self-sufficiency.

Unfortunately, a closer investigation reveals an antinomy. If consciousness is itself the cause for the consciousness-of-consciousness, then what allows the antecedent consciousness to be conscious? This reflection of one consciousness against another, a \emph{dyadic} relationship -- is absolutely unbounded in its progression:

\blockcquote[11]{Sartre}{
    If we accept the law of the knowing-known dyad, a third term will become necessary for the knowing in its turn to become known, and we are placed in a dilemma. Either we stop at some term within the series \ldots\ -- in which the phenomenon in its totality collapses into the unknown (i.e., we always come up against a reflection that is not conscious of itself and is the final term) -- or we declare an infinite regress to be necessary, which is absurd.
}

\noindent
As Sartre himself recognised, if we attempt to bound this progression, our limit is arbitrary and externally imposed: \textquote{the phenomenon in its totality collapses into the unknown}. And if we do not constrain this regression, we achieve an infinite regression, a circularity. Like the physicist, we had tried to \enquote{balance the terms} of our equation by changing a sign, positing a reflective particle-antiparticle pair. But very quickly, we realised that such parameters leads to asymptotic growth, an unbounded meta-physical binding-energy. This particle that we posit is unstable since it requires infinity, and as a result it cannot exist in our ontological cosmology.

How can we account for the consciousness-of-consciousness then? To begin, our previous exercise has demonstrated that consciousness cannot be dyadic: we cannot split the being-of-consciousness into two symmetrical components. Likewise, through a proof via induction, we can also demonstrate that consciousness cannot be split into \emph{any} number of components -- the being-of-consciousness must be unitary. Is it possible to account for our consciousness-of-consciousness in a unitary manner, where the very nature of our being-of-consciousness contains the \enquote*{self-consciousness} which we seek? Sartre explores this possibility through the meta-physical thought experiment of reflection: We reflect upon our consciousness -- imagine being conscious of a feeling, a desire -- whatever object that through your awareness, you (i.e. your consciousness) \emph{posit} towards. Now imagine reflecting upon that [reflected] consciousness from a moment ago: you think about your consciousness of desire, of the object:

\blockcquote[11]{Sartre}{
    In the act of reflection, I bring judgements to bear on my reflected consciousness; I am ashamed of it, I am proud of it, I want it, I reject it, etc. The immediate consciousness that I have of perceiving does not allow me either to judge, or to want, or to be ashamed. It does not \emph{know} my perception, or \emph{posit} it: all that is intentional within my current [act of] consciousness is directed outward, towards the world.
}

\noindent
What does Sartre discern, from the fact that our original consciousness of perception (the immediate consciousness from the above passage) does not contain any \enquote*{thing} which allows our subsequent judgement? It is a subtle observation that recognises how our subsequent reflecting consciousness, does not \emph{contain} the judgements which it invokes on the reflected consciousness. To elucidate this further, consider the steps of Sartre's demonstration carefully: Our subsequent reflecting consciousness is a positional, thetic consciousness. The thetic object of the reflecting consciousness, that which it \emph{posits} towards, is the original reflected consciousness (that feeling, or desire, as per our thought experiment). And of course, the original reflected consciousness has that feeling, or desire, as it's thetic object (that which it posits towards). Nothing in this above progression \emph{contains} the judgements themselves. That shame, that pride, that desire, that rejection -- all of these judgements yielded by the act of reflection is nowhere to be found in the positional objects of either the reflecting consciousness, nor the original reflected consciousness. This thought experiment of the reflection can be represented in more rigorous information-theoretic forms, as the formal data-structure of an \emph{linked list}.\footnote{A linked list is a form of data-structure in Computer and Information Science. Linked lists contain nodes, each having a data and a reference. The parallels between Computer Science and Ontology are deeply intriguing, for the former is almost a practical implementation of the latter.} And it will likewise demonstrate the absence of these judgements, which are manifestations of the consciousness-of-consciousness, which every act of thetic positional consciousness necessarily contains.

What conclusions can the metaphysician draw from this ontological thought experiment? It is a demonstration of the inherence of our consciousness-of-consciousness, which is present in every act of positional consciousness that we take. 
\blockcquote[11]{Sartre}{This spontaneous consciousness that I have of my perception is \emph{constitutive} of my perceptual consciousness. In other words, any positional consciousness of an object is at the same time a non-positional consciousness of itself.} Hence, we may derive the following conclusion. The being-of-consciousness is not only a thetic, positional consciousness, but it is also a \emph{non-thetic}, non-positional consciousness. To use Sartre's terminology, this is the \emph{pre-reflective cogito} of consciousness, the \textcquote[11]{Sartre}{immediate and non-cognitive relationship of self-to-self.} This is not a dichotomy or dualism, we have not re-introduced the dyad which was demonstrated to fail. But rather, to quote Sartre:

\blockcquote[12]{Sartre}{We can express this [the nature of consciousness] in these terms: any conscious existence exists as the consciousness of existing. We can understand now why the most basic consciousness of consciousness is not positional: because it and the consciousness of which it is conscious \emph{are one and the same}. In a \emph{single movement}, consciousness determines itself as consciousness of perception, and as perception.}

\noindent
Thus, we have completed our account of the ontology of consciousness at this stage of our metaphysical cosmogony. This definition of the being-of-consciousness possesses the following characteristics: It is a contentless, positional being, that posits towards a thetic object. This object of consciousness is by necessity outside of consciousness, hence the being-of-consciousness is transcendent. However, this act of thetic positional consciousness is one and the same with a certain \enquote*{self-consciousness}, properly defined as a pre-reflective cogito that is constitutive and unitary with the thetic act of positing itself. Thus, we have fulfilled all the terms that our equation requires -- the metaphysician at her blackboard is able to balance her ontological particle. Now with a solid understanding of consciousness, we are able to examine the relationship between phenomena and consciousness. We proceed onwards, with the next step of our ontological deduction.

\section{Attempts to Derive Phenomena from Consciousness}

Let us now step back, and take an inventory of our incipient ontological theory. The metaphysician sets aside her chalk, and glances up at the propositions of her metaphysical system. What are the elements available to us within our domain of discourse? We began with phenomena, the raw sensory data of our existence. We separated phenomena into the phenomena-of-being, and the being-of-phenomena, the latter which we seek. Furthermore, we realised that the being-of-phenomena points towards consciousness, -- so setting aside the being-of-phenomena for the moment, we embarked on an investigation of the being-of-consciousness. Now, after a careful series of demonstrations and thought experiments, we reach a clear and well-defined definition for the being-of-consciousness itself. However, it is important for us to keep in mind that the \emph{only} well-defined aspect of our ontology at this point is the being-of-consciousness. Our elucidation of the being-of-phenomena is provisional and incomplete -- we are only able to state that it points us towards consciousness, but not of its relationship at all. However, now that we understand the being-of-consciousness, we can attempt to complete the inequality -- and see how the being-of-consciousness relate to the being-of-phenomena, which we seek.

% For the next section, I will need to to present the 'big bang' of Sartrean ontology. What is the best narrative
% structure to accomplish such a task? I think I'll do the following:
% Begin by talking about how right now, the only well-defined and known ontological being is the being of our
% consciousness. In this world of consciousness, there is nothing else.
% Next, talk hypothetically about how consciousness can possibly arise to phenomena. Phenomena does exist -- we see all
% sorts of ontological evidence of it, in the aftermath of its genesis. This is the ontological cosmic background
% radiation that we talked about earlier, after all.
% However, tracing back the path of phenomena towards being, we are left with the same question. How can phenomena arise
% from the being-of-consciousness? (Remember, at this point all you have is the being-of-consciousness. 
% The in-itself and for-itselfs are not defined yet!) 
% This is the time to make a few (abortive) attempts to derive the being-of-phenomena from the being-of-consciousness.
% Trace Sartre's progression, write about how both the being-of-phenomena and the being-of-consciousness are transcendent
% However show that ultimately it is NOT possible to derive being-of-phenomena from the being-of-consciousness through
% any positive act!
% This is how you present the derivation of being as a negative act. Perhaps briefly at first, then use the big-bang
% metaphor, and drive it on towards completion.

% Some additional things to make sure to keep in mind and explore
%  - It is this derivation of being as a negative act through which we first see contingency enter the mix.
%  - Contingency is present in the nature of self-presence. In self presence, by having this seperation of nothingness
%  - of consciousness to its own being, we see the determination of the in-itself and the for-itself.
%  - Make sure to explore self-presence thoroughly.

\section{Consciousness and Nothingness}
% This is the part where we elaborate on how nothingness is the relationship between consciousness and phenomena, how this negative act is the 'baryogenesis' of consciousness


%  2. Being for-itself
%      1. We'll need to begin with an examination self-presence, as the 'Baryon asymmetry' of ontology 



%  3. Being for the Other
%      1. We'll have to examine how without the other, the world is merely 'psychological'


\chapter{The Contingency of Human Existence}

% Make sure to remember to talk about the following concept: the 'ontological-complexity' of a metaphysics. Where there is roughly a balance to be had between solipsism, and nihilism, but a complex array of options between the two which defines ontological complexity:
% 
% Solipsism - the ontology is too simple, all is reduced to one
% Finite ontology - the ontology of chess and checkers, a finite universe.
% Deterministic ontology -- the ontological complexity expands at a constant rate
% A Free ontology - the ontology expands non-linearly, but just right -- there is freedom
% Nihilism - the ontology is too general, there is no being at all

\chapter{The Teleology of Contingency}

% What is the relationship between existentialist ontology and morality? And philosophy? Are these things possible at all?

\chapter{Conclusion}
