\chapter{The Cosmogony of Sartrean Ontology}
% Explain briefly how the question of contingency comes from nothingness-- hence in order for us to explore this question properly, we need to first understand where nothingness comes from 

% Begin an explanation of the genesis of contingency from nothingness, using a physical analogy of Cosmogony.
% We're basically using the literary device of an extended metaphor with (physical) cosmogony, of an 'ontological Big-Bang' to set up a ordered exposition of Sartre's ontology.

What is contingency? What is the nothingness inside our being, and how is the relationship between our being and nothingness a relationship of contingency? How did contingency arise in the first place? In order to answer these questions, which are the more subsequent propositions of Sartre's ontology -- we must first take a look at nothingness, and begin with the postulates and definitions of Sartre's system. We must look at how being emerges in the first place -- and trace our way through the history of its genesis in which the ontology of our human-reality emerges. This process is an ontological cosmogony, and the progression of our science shares familiar motifs with its counterpart in physical cosmology: complete with its own epochs and symmetry-breaking. This presentation will be an analytic overview of Sartre's ontology, a base-camp that we will establish in the pursuit of our question. Hence, we will begin our inquiry into the contingency of our being, with the very cosmogony of Sartrean ontology itself.

\section{The Foundation of Phenomena}
% This extended metaphor will take us through the first three major parts of Sartre's Being and Nothingness:
%  1. The Problem of Nothingness -- where it comes from
%      1. First, we begin with existents -- and the phenomena which we perceive
%      2. We acknowledge the 'factual neccessity' of there being a consciousness to perceive phenomena
%      3. Next, we present the problem where phenomena must have its own being independent of the being of conscousness
%      4. We present the 'Baryogenesis' of ontology, where how the negative act of consciousness begets objectivity.

Our investigation begins from the foundation of phenomena -- the basic realm of derivative ontological data that is readily accessible to our metaphysics. We happen to live in a world of phenomena -- a rich plenum of perceptions that forms the infinite state-space of human-reality.\footnote{\emph{Human-reality} is similar to the \emph{Dasein} of Heidegger, although we will use it in the context of Sartre's phenomenology -- hence avoiding the original German.} How can we find being, starting from the raw data of the phenomena? Very quickly, we realise that there is a distinction between the \emph{phenomenon-of-being}, and the \emph{being-of-phenomena} -- at least, a distinction that is possible in the infancy of our incipient ontology. What we seek to grasp is the \emph{being-of-phenomena} -- the universal, ontological basis for all phenomena. In contrast, the phenomenon-of-being is the more superficial \emph{appearance} of any arbitrary being -- much more accessible to us, but not necessarily the same as the \emph{being-of-phenomena}. \textcquote[6]{Sartre}{Is the being that is disclosed to me, that \emph{appears} to me, the same in nature as the being of the existents that appear to me?} Can the former (the phenomena-of-being), be reduced to or otherwise lead us to the latter (the being-of-phenomena)? 

Unfortunately, such a reduction is not possible -- we cannot reduce the subsequent phenomena-of-being to the more fundamental being-of-phenomena. The being of an object cannot come from the object itself: \textcquote[6]{Sartre}{it is not possible, for example, to define being as a \emph{presence}, since \emph{absence} also discloses being, since not being \emph{there} is still a way of being.} This impossibility of reducing the being-of-phenomena to the phenomena-of-being is explored rigorously by Sartre in his introduction, for even more sophisticated attempts at such a reduction is ultimately an appeal to \emph{knowledge} as a foundation for being -- with knowledge necessarily defined as the simple ratio or proportionality between an existent's being and it's appearing, i.e. it's phenomena \autocite[7]{Sartre}. Hence, the being-of-phenomena is by necessity separate and irreducible from the phenomena-of-being -- which is to say that the being of phenomena lies \emph{outside} of the object of phenomena itself. \textcquote[7]{Sartre}{In brief, the phenomenon of being is \enquote{ontological} in the sense in which Saint Anselm's and Descartes's proof is called ontological. It is a call for being: it requires, insofar as it is a phenomenon, a transphenomenal foundation.} Hence, Sartre concludes that the being of phenomena is \emph{transphenomenal}: i.e. outside the phenomena.\footnote{This is not to say that phenomena is somehow dualistic. As Sartre carefully explains in the introduction, the being of an existent is in \textcquote[1]{Sartre}{the series of appearances that manifest it.} The being of an object is its phenomena -- but here we are inquiring after the metaphysics of phenomena itself.}

% TODO: Remember to add a footnote here to justify how this transphenomenality of the being-of-phenomena is not a dualism.

If the being-of-phenomena is transphenomenal -- then where can being lie? What external \enquote*{thing}\footnote{Single quotes, i.e. scare quotes, are used to designate concepts provisionally or rhetorically.} can serve as the foundation for the being-of-phenomena? \emph{Another} being is the only choice that is available for us -- for in this stage of the development of our ontological theory, the only two particles of our metaphysics are being and phenomena. Phenomena cannot be the foundation of phenomena -- for that would lead to an infinite regression, a circularity. Hence, only another being can be the basis for phenomena's own being. What is this other being, which serves as the condition for phenomena? Or in other words, if phenomena is merely the appearance of being, but not the being itself -- \emph{then to whom does phenomena appear to?} This being is \emph{consciousness} -- the being \emph{to whom} there are appearances in the first place. And thus, we posit the first new elementary particle of our ontological cosmology. In our survey of the background of phenomena, we discover a being-of-phenomena which has no immanent source -- but a transphenomenal origin. This transphenomenal being-of-phenomena points us towards consciousness, like how the cosmic microwave background radiation of the physical universe points us towards the Big Bang. Hence, as with both -- in order to understand the former, we must investigate the latter. At this stage we cannot say anything about the relationship between the being-of-the-phenomena and consciousness -- indeed, we know nothing about the ontological structure of consciousness itself. But now we have a direction for our inquiry, and a method not-too-dissimilar from an astrophysicist peering back into the progression of the Big Bang. We trace phenomena towards consciousness, and proceed to investigate the ontology of consciousness itself.\footnote{Here I actually depart slightly from the original progression of Sartre's rhetoric. Sartre first explores knowledge in further detail, as knowledge is a more direct embodiment of the relationship between the being-of-phenomena and the phenomena-of-being. It is only after establishing the ontology of knowledge, does Sartre then turn towards the \emph{knower} of the knowledge, which leads us to consciousness. The progression towards knowledge first, then consciousness -- is a necessity of the synthetic nature of presenting a new metaphysics \emph{ab initio}. Thankfully, as his descendents, we may present his ideas analytically.}

\section{The Ontology of Consciousness}

% Talk about the factual neccessity of consciousness

What is the ontology of consciousness? 
% And how does consciousness relate to the being-of-the-phenomena? Does consciousness \enquote*{give} phenomena being, or perhaps \enquote*{generate} being for phenomena? Phenomenon leads us towards consciousness, but what is their exact relationship?
Are we certain that there is such a being as consciousness at all? To begin, it does not appear that the being of consciousness is certain -- for while we assert that \enquote{phenomenon must appear to a being}, such an assertion does not seem self-evident. After all, it is possible as a thought experiment for the metaphysician to imagine a world with \enquote*{being}, but without consciousness. But such a world would also fundamentally be without phenomena -- there is \enquote*{being} but no appearance. And further yet, the ontology of this hypothetical world is unstable, for the so-called \enquote*{being} postulated cannot derive its foundation from anywhere. This lemma is presented by Sartre when he states that \textcquote[14]{Sartre}{consciousness is not \emph{possible} before being but instead comprises -- in its being -- the source and condition of all possibility, its existence implies its essence. This is felicitously expressed by Husserl as its [i.e. consciousness's] \enquote{factual necessity.}} This factual necessity is akin to a certain \emph{anthropic principle}\footnote{The \emph{anthropic principle} refers to the biases which favour the existence of an observer, for the fact that we exist to observe in the first place. The term is borrowed from physical cosmology.} of ontology. Consciousness does not necessarily have to exist, but the fact that it exists makes its non-existence inconceivable. As we live in a world with consciousness, we can accept this factual necessity as granted -- it is the axiom of our metaphysics. There may be other metaphysics for worlds without consciousness, but such speculative ontology is beyond the scope of our work.

Thus, Sartre asserts the existence of the \emph{being-of-consciousness}. But we have not yet understood the relationship between the being-of-consciousness and the being-of-phenomena yet. What is the relationship between phenomena and consciousness? Is it not the case that phenomena \emph{appears} to consciousness? Can we claim that these appearances are held within consciousness, in the same way that we would speak of \enquote{becoming conscious of \enquote{something?}} Sartre asserts that consciousness is fundamentally \emph{positional}. \textcquote[9]{Sartre}{All consciousness is consciousness of something. In other words, there is no [act of] consciousness that does not \emph{posit} a transcendent object, or if you prefer, consciousness has no \emph{content.}} 
This contentless nature of consciousness is significant, for it clarifies that this act of positing is not like the admixture of two beings, but only a grasping of what consciousness posits towards.
\textcquote[10]{Sartre}{Consciousness is positional in that it transcends itself to reach an object and is exhausted by this act of positing.} This positional nature of consciousness is also the reason why the ontology of consciousness is a \emph{transcendence}: \textcquote[22]{Sartre}{transcendence is a constitutive structure of consciousness, which is to say that consciousness is born \emph{bearing on a being that it is not.}} Borrowing from Husserl's phenomenological vocabulary, consciousness in this positional sense is also referred to as \emph{thetic}: pertaining to a \emph{thesis}, an object of consciousness which we have \emph{posited} towards.

% Talk about the twofold nature of consciousness as both positional and pre-cognitive

Now we gain an account of consciousness which ascribes to its ontology the following characteristics: It is contentless, and positional. It posits towards a being that is outside the being-of-consciousness itself, a \emph{transcendence}. Is this account of consciousness ontologically complete, or even sufficient for our purpose? Can we now begin to explicate the being-of-phenomena, from this hypothesised being-of-consciousness? Not yet. For this provisional consciousness of ours, with the above parameters and terms, fails to account for one important (and perhaps even definitive) attribute of consciousness: we are conscious \emph{of} our consciousness. What does it mean to be conscious of our consciousness? \textcquote[10]{Sartre}{The necessary and sufficient condition for a knowing consciousness to be knowledge of its object, is that it should be conscious of \emph{itself} as being this knowledge.} Our thetic, positional consciousness satisfies the first half of this criterion. Our provisional consciousness posits towards its object\footnote{At this stage of our ontology's development, the phrase \emph{object} must contain the bare minimum of ontological significance. When we refer to an \enquote{object of consciousness} in these cases, we strictly mean the grammatical object: the being that which consciousness posits towards. Concepts such as \emph{objectivity} are strictly undefined now, and can only be derived later.}, which we give the name \emph{knowledge}. However, we have failed to account or provision a means for which our consciousness is \textquote{aware of itself as being this knowledge.} To quote Sartre on the strict necessity of this condition:

\blockcquote[10]{Sartre}{%
    If my consciousness were not consciousness of being conscious of a table, it would hereby be conscious of the table without being conscious that it was so, or, alternatively, it would be a consciousness that did not know itself, an unconscious consciousness -- which would be absurd.
}

\noindent
Hence, in order to elucidate the relationship between the being-of-phenomena and the being-of-consciousness, we must first complete our account of the being-of-consciousness with an account for this second term. Like the physicist at the blackboard, we discover that our initial equation does not add up to the sum of the particle which we observe. How can we account for this \enquote*{self-conscious}\footnote{Although the phrase self-consciousness is a more succinct term for the consciousness-of-consciousness, we will refrain from using it -- both in order to stay faithful to Sartre's own terminology, as well as to avoid any confusion with \emph{self-presence}, a concept which we will elaborate in later sections.} component of consciousness? Or using Sartre's terminology, this \emph{consciousness-of-consciousness}? The first and theoretically simplest method, is for us to appeal to \emph{reflectivity} as a foundation for this consciousness-of-consciousness. Consciousness is positional. Why not allow consciousness to be conscious of itself? In this case, the thetic object of consciousness would \emph{be} consciousness -- our consciousness posits towards our consciousness, in other words. This approach is simple, but not improperly so -- after all, it possesses a theoretical elegance, a certain balance and self-sufficiency.

Unfortunately, a closer investigation reveals an antinomy. If consciousness is itself the cause for the consciousness-of-consciousness, then what allows the antecedent consciousness to be conscious? This reflection of one consciousness against another, a \emph{dyadic} relationship -- is absolutely unbounded in its progression:

\blockcquote[11]{Sartre}{
    If we accept the law of the knowing-known dyad, a third term will become necessary for the knowing in its turn to become known, and we are placed in a dilemma. Either we stop at some term within the series \ldots\ -- in which the phenomenon in its totality collapses into the unknown (i.e., we always come up against a reflection that is not conscious of itself and is the final term) -- or we declare an infinite regress to be necessary, which is absurd.
}

\noindent
As Sartre himself recognised, if we attempt to bound this progression, our limit is arbitrary and externally imposed: \textquote{the phenomenon in its totality collapses into the unknown}. And if we do not constrain this regression, we achieve an infinite regression, a circularity. Like the physicist, we had tried to \enquote{balance the terms} of our equation by changing a sign, positing a reflective particle-antiparticle pair. But very quickly, we realised that such parameters leads to asymptotic growth, an unbounded meta-physical binding-energy. This particle that we posit is unstable since it requires infinity, and as a result it cannot exist in our ontological cosmology.

How can we account for this consciousness-of-consciousness then? To begin, our previous exercise has demonstrated that consciousness cannot be dyadic: we cannot split the being-of-consciousness into two symmetrical components. Likewise, through a proof via induction, we can also demonstrate that consciousness cannot be split into \emph{any} number of components -- the being-of-consciousness must be unitary. Is it possible to account for our consciousness-of-consciousness in a unitary manner, where the very nature of our being-of-consciousness contains the \enquote*{self-consciousness} which we seek? Sartre explores this possibility through the meta-physical thought experiment of reflection: We reflect upon our consciousness -- imagine being conscious of a feeling, a desire -- whatever object that through your awareness, you (i.e. your consciousness) \emph{posit} towards. Now imagine reflecting upon that [reflected] consciousness from a moment ago: you think about your consciousness of desire, of the object:

\blockcquote[11]{Sartre}{%
    In the act of reflection, I bring judgements to bear on my reflected consciousness; I am ashamed of it, I am proud of it, I want it, I reject it, etc. The immediate consciousness that I have of perceiving does not allow me either to judge, or to want, or to be ashamed. It does not \emph{know} my perception, or \emph{posit} it: all that is intentional within my current [act of] consciousness is directed outward, towards the world.
}

\noindent
What does Sartre discern, from the fact that our original consciousness of perception (the immediate consciousness from the above passage) does not contain any \enquote*{thing} which allows our subsequent judgement? It is a subtle observation that recognises how our subsequent reflecting consciousness, does not \emph{contain} the judgements which it invokes on the reflected consciousness.
% To elucidate this further, consider the steps of Sartre's demonstration carefully:
To clarify Sartre's demonstration, let us consider his steps in more detail:
Our subsequent reflecting consciousness is a positional, thetic consciousness. The thetic object of the reflecting consciousness, that which it \emph{posits} towards, is the original reflected consciousness (that feeling, or desire, as per our thought experiment). And of course, the original reflected consciousness has that feeling, or desire, as it's thetic object (that which it posits towards). Nothing in this above progression \emph{contains} the judgements themselves. That shame, that pride, that desire, that rejection -- all of these judgements yielded by the act of reflection is nowhere to be found in the positional objects of either the reflecting consciousness, nor the original reflected consciousness. This thought experiment of the reflection can be represented in more rigorous information-theoretic forms, as the formal data-structure of an \emph{linked list}.\footnote{A linked list is a form of data-structure in Computer and Information Science. Linked lists contain nodes, each having a data and a reference. The parallels between Computer Science and Ontology are deeply intriguing, for the former is almost a practical implementation of the latter.} And it will likewise demonstrate the absence of these judgements, which are manifestations of the consciousness-of-consciousness, which every act of thetic positional consciousness necessarily contains. Hence, our current account of the ontology of consciousness is inadequate, for it does not agree with the raw data of our own human experience.

What conclusions can the metaphysician draw from this ontological thought experiment? It is a demonstration of the inherence of our consciousness-of-consciousness, which is present in every act of positional consciousness that we take. 
\blockcquote[11]{Sartre}{This spontaneous consciousness that I have of my perception is \emph{constitutive} of my perceptual consciousness. In other words, any positional consciousness of an object is at the same time a non-positional consciousness of itself.} Hence, we may derive the following conclusion. The being-of-consciousness is not only a thetic, positional consciousness, but it is also a \emph{non-thetic}, non-positional consciousness. To use Sartre's terminology, this is the \emph{pre-reflective cogito} of consciousness, the \textcquote[11]{Sartre}{immediate and non-cognitive relationship of self-to-self.} This is not a dichotomy or dualism, we have not re-introduced the dyad which was demonstrated to fail. But rather, to quote Sartre:

\blockcquote[12]{Sartre}{%
    We can express this [the nature of consciousness] in these terms: any conscious existence exists as the consciousness of existing. We can understand now why the most basic consciousness of consciousness is not positional: because it and the consciousness of which it is conscious \emph{are one and the same}. In a \emph{single movement}, consciousness determines itself as consciousness of perception, and as perception.
}

\noindent
Thus, we have completed our account of the ontology of consciousness at this stage of our metaphysical cosmogony. This definition of the being-of-consciousness possesses the following characteristics: It is a contentless, positional being, that posits towards a thetic object. This object of consciousness is by necessity outside of consciousness, hence the being-of-consciousness is transcendent. However, this act of thetic positional consciousness is one and the same with a certain \enquote*{self-consciousness,} properly defined as a pre-reflective cogito that is constitutive and unitary with the thetic act of positing itself. Thus, we have fulfilled all the terms that our equation requires -- the metaphysician at her blackboard is able to balance her ontological particle. This is the ontology of the being-of-consciousness. Only now with this understanding, are we able proceed, and examine the relationship between phenomena and consciousness.

% For the next section, I will need to to present the 'big bang' of Sartrean ontology.

\section{The Big Bang of Sartrean Ontology}

% Begin by talking about how right now, the only well-defined and known ontological being is the being of our
% consciousness. In this world of consciousness, there is nothing else.

Let us now step back, and take an inventory of our incipient ontological theory. The metaphysician sets aside her chalk, and glances up at the propositions of her metaphysical system. What are the elements available to us within our domain of discourse? We began with phenomena, the raw sensory data of our existence. We separated phenomena into the phenomena-of-being, and the being-of-phenomena, the latter which we seek. Furthermore, we realised that the being-of-phenomena points towards consciousness, -- so setting aside the being-of-phenomena for the moment, we embarked on an investigation of the being-of-consciousness. Now, after a careful series of demonstrations and thought experiments, we reach a clear and well-defined definition for the being-of-consciousness itself. 

% However, it is important for us to keep in mind that the \emph{only} well-defined aspect of our ontology at this point is the being-of-consciousness. Our elucidation of the being-of-phenomena is provisional and incomplete -- we are only able to state that it points us towards consciousness, but not of its relationship at all. However, now that we understand the being-of-consciousness, we can attempt to complete the inequality -- and see how the being-of-consciousness relate to the being-of-phenomena, which we seek.

% Next, talk hypothetically about how consciousness can possibly arise to phenomena. Phenomena does exist -- we see all
% sorts of ontological evidence of it, in the aftermath of its genesis. This is the ontological cosmic background
% radiation that we talked about earlier, after all.
% However, tracing back the path of phenomena towards being, we are left with the same question. How can phenomena arise
% from the being-of-consciousness? (Remember, at this point all you have is the being-of-consciousness. 
% The in-itself and for-itselfs are not defined yet!) 
% This is the time to make a few (abortive) attempts to derive the being-of-phenomena from the being-of-consciousness.
% Trace Sartre's progression, write about how both the being-of-phenomena and the being-of-consciousness are transcendent
% However show that ultimately it is NOT possible to derive being-of-phenomena from the being-of-consciousness through
% any positive act!

% To begin, is it possible for us to \enquote*{derive} the being-of-phenomena from the being-of-consciousness? In this manner, would it be possible for us to state that consciousness \enquote*{generates} the being-of-phenomena, from within itself? In any of these cases, we are essentially attempting to reduce the being-of-phenomena to the being-of-consciousness itself -- the act of derivation or generation ultimately attributes the cause of the phenomena's being to the being-of-consciousness. This reduces the being-of-phenomena to knowledge. There are two possible means in which such a reduction can be achieved: through either a relationship of \emph{relativity}, or a relationship of \emph{passivity}. Relativity posits a definite relationship between the being-of-phenomena and the being-of-consciousness, with phenomena serving as a junior partner in the personal union. However, such a relativity does not reject the being of the being-of-phenomena itself, hence relativity alone is insufficient for the reduction. \textcquote[18]{Sartre}{Relativity does not excuse us from the need to examine the \emph{percipi}'s being.} The only other possibility is to reduce phenomena to a passivity -- where like an empty vessel, its being is given by the being-of-consciousness. Unfortunately, Sartre also demonstrates the impossibility of this reduction:

% \blockcquote[18]{Sartre}{[Passivity] is a relation between one being and another being, and not between a being and a nothingness. It is impossible for the \emph{percipere} [being-of-consciousness] to assign the \emph{perceptum} [phenomena] its being, because in order to be assigned anything, the \emph{perceptum} would already need to be given in some way and thus to exist before receiving its being.}

% \noindent
% Hence, we are forced to conclude that the being-of-phenomena is cannot be derived from the being-of-consciousness. This is a puzzling conclusion. What is the being-of-phenomena, and where does it come from? To answer this, let us take a closer look at the universe of the being-of-consciousness itself.

What is the being-of-consciousness? It is a thetic, positional consciousness \emph{of} some impression. But more importantly, every act of consciousness is also an absolute non-thetic precognitive awareness of itself, as the consciousness-of-consciousness. This unitary \enquote*{self-consciousness,} the awareness of the self in every act of positioning -- is the source of our \emph{subjectivity.} \textcquote[21]{Sartre}{Consciousness is a real subjectivity, and an impression is a subjective plenitude.} This subjective plenitude refers to our absolute access and unity with the object of consciousness. When I am conscious of a sensation, every part of that sensation is available to me -- there is nothing \enquote*{hidden} or inaccessible behind that sensation.
\textcquote[24]{Sartre}{The phenomenon of being, like any basic phenomenon, is disclosed immediately to consciousness \ldots\ what Heidegger calls a \enquote{preontological understanding.}} 
% TODO: Perhaps add an elaboration on how sensation is always accessible, through an example?
That sensation simply \emph{is} a part of me. No part of what I am conscious of can be concealed from me, for the very act of positing towards an object of consciousness also necessarily implies my own consciousness of my consciousness. This was the conclusion of our earlier demonstration, and any rejection of it would result in an \enquote{unconscious consciousness,} an absurdity. From this world of positive, pure subjectivity, there can be no separation from the self, no objectivity. Everything is simply one with the self, an absolute solipsism of being.

\noindent
What does this sort of world look like? Let us engage in an act of speculative metaphysics, for the sake of illustration alone. This is the ontology of the Point, a one-dimensional universe in the world of Edwin Abbott's novel \emph{Flatland}. When the two-dimensional denizens of Flatland visit \enquote{Point-land}, they enter a one-dimensional world with neither distance nor separation. Here there is only a single Point -- who is the sole inhabitant, monarch, and universe in one. The Point is a being-of-consciousness who embodies this subjective plenitude in full: he perceives all things, even the attempts at communication by the Flat-landers, as thoughts originating within his own consciousness. There is no objectivity, no externality -- only self. This short vignette is a surprisingly evocative demonstration of the metaphysics of a world without separation or distance, a world without \emph{nothingness.}

With that thought, our Metaphysician picks up her instruments, and looks back upon our blackboard. How can we prevent a similar fate for our own ontology? What must we do, to allow the existence of an independent, \emph{objective} world, where appearances are \textcquote[4]{Sartre}{connected by a principle that does not depend on my whim?} 
%How can we bring about the being-of-phenomena from this being-of-consciousness?
What can allow us to derive the rich, phenomenal world of our human-reality, from this suffocating plenitude of the self? A metaphysics with only one particle is a static one, a sterility. There has to be a \emph{force} which brings about change. Thus, we reach the core of Sartre's ontological thesis -- his metaphysical discovery. It is through \emph{nothingness} which we find the being-of-phenomena. Specifically, the foundation of our objective, phenomenal world -- comes from the being-of-phenomena, \emph{as the non-being of the being-of-consciousness.}

\blockcquote[21]{Sartre}{[If] we want the phenomenon's being to depend on consciousness, the object will need to distinguish itself from consciousness not through its \emph{presence}, but through its \emph{absence}, not through its plenitude but through its nothingness. If being belongs to consciousness, the object must differ from consciousness not insofar as it is another being but insofar as it is a \emph{non-being}.}

\noindent
How does this look like? What does it mean for an object to \enquote{differ from consciousness, as a non-being?} Here is an image that we can use to understand the full force of Sartre's demonstration. Let us begin with the world of the being-of-consciousness, in its full, undiluted subjectivity. What is this world? It is a singular world, a pure subjectivity. It is the ceaseless mantra of \enquote{\emph{\ldots\ I am, I am, I am, I am \ldots,}} endless in its totality. This universe is isomorphic and homogenous -- and any object that attempts to derive its being from this universe will instantly lose itself in this ceaseless mantra, becoming one with the \enquote{\emph{I am}} of subjectivity. After all, the very being of consciousness is its own \enquote*{self-consciousness.} Hence, the only way for an object to exist at all, as something \emph{separate} and apart from consciousness -- is as a non-being of consciousness, a defiant \enquote{\emph{I am not.}} Or in Sartre's words: \textcquote[21]{Sartre}{To be conscious of something is to confront a full and concrete presence that \emph{is not} consciousness.} Only through its being as a non-being-of-consciousness, does the ceaseless mantra of \enquote{\emph{I am, I am, I am, \ldots}} get interrupted, by something that consciousness \enquote{\emph{am not.}} The being-of-consciousness is like a luminiferous aether, which reaches and spreads and is only stopped by the dense, opaque nothingness of what it is not. 

Thus, we discover the ontology of the being-of-phenomena. The being-of-phenomena is in relation to the being-of-consciousness, not through a generation or derivation, or indeed any positive action at all. But rather, \emph{the being-of-phenomena is a non-being of the being-of-consciousness.} This nothingness at the heart of the being-of-phenomena is the chief ontological innovation of Sartre's phenomenological ontology. Nothingness is a fundamental force, the addition of which triggers a rapid expansion of our metaphysics, an ontological Big Bang. From the stable, singular solipsism of pure subjectivity, we achieve a baryogenesis. All of a sudden, the sterile plenitude of our subjective universe is struck by an ontological inflation, yielding a world of metaphysical matter, of the being-of-the-phenomena. This is the cosmogony of Sartrean ontology, -- the birth of \emph{phenomena} in the universe. And with this baryogenesis, we also reach a crucial \emph{asymmetry} in our metaphysics -- the introduction of \emph{contingency} into our being.

% This is how you present the derivation of being as a negative act. Perhaps briefly at first, then use the big-bang
% metaphor, and drive it on towards completion. We elaborate on how nothingness is the relationship between consciousness and phenomena, how this negative act is the 'baryogenesis' of consciousness
% Some additional things to make sure to keep in mind and explore
%  - It is this derivation of being as a negative act through which we first see contingency enter the mix.
%  - Contingency is present in the nature of self-presence. In self presence, by having this seperation of nothingness
%  - of consciousness to its own being, we see the determination of the in-itself and the for-itself.
%  - Make sure to explore self-presence thoroughly.

\section{Being-\emph{for}-Itself, Being-\emph{in}-Itself, and Self-Presence}

% Previous draft, not sure if I want to keep:
% Let us pause for a moment, and survey the state of our metaphysics at this point. The Metaphysician looks back upon the blackboard, the results of our ontological investigations. Sartre's nothingness has inflated the original, subjective singularity of the being-of-consciousness, triggering the cosmological inflation which yields phenomena. Now the universe of our metaphysics has expanded, there is both being-of-consciousness, and being-of-phenomena. The being-of-phenomena is a negation of the being-of-consciousness, specifically a non-being of the being-of-consciousness. But what does it mean for there to be a non-being, of the being-of-consciousness? Our previous, preliminary definition of consciousness contains two characteristics -- a positional, thetic component towards an object, as well as a non-thetic, pre-reflective cogito, towards the self. For us to say the being-of-phenomena is a non-being of the being-of-consciousness is an elucidation of the thetic, positional aspect of consciousness -- but through this elaboration, our understanding of the ontology of consciousness has changed. What is different, now that we accept nothingness as a fundamental force within our metaphysics?

What happened here, now that we introduced nothingness into our metaphysics? What does it mean to say that we have introduced contingency into our being?  For our consciousness to be thetic and positional, it must posit towards an object, a phenomenon -- that is not itself. This is what Sartre means, when he says \textcquote[10]{Sartre}{consciousness is born bearing on a being that it is not.} Hence, every act of positional consciousness is an act of self-negation, a \emph{questioning}. Every time our consciousness posits towards an object, it is asking: \enquote{Is this being, \emph{my} being?}\footnote{Sartre's actual demonstration is more nuanced: the very act of questioning necessitates a negation -- for to ask of a being is to allow the possibility of its non-being.} This is how Sartre arrives at his definition of the being-of-consciousness, as a being which contains within itself, a question of its own being: \textcquote[23]{Sartre}{Consciousness is a being for whom in its  being there is a question of its being, insofar as this being implies a being other than itself.} But what sort of being is the kind those very ontology contains \textquote{a question of its [own] being?} Right now, this is a puzzling question, it may not seem important or particularly essential at first glance. However, as we follow the question along its path, we will reveal the first fundamental asymmetry of our metaphysical cosmology: that of the \emph{being-for-itself}, and the \emph{being-in-itself.}

Let us begin our investigation. \enquote{What sort of being, is one whose very ontology, \emph{is} a question of its own being?}\footnote{The obscurity of this sentence is due to my avoidance of using words like \enquote*{nature}, or \enquote*{meaning,} in fear of introducing ontological confusions. A more casual formulation of the same question can read as: \enquote{What is the nature of a being, that is defined by a question of its own being?}} It's clear that consciousness belongs to this category of beings -- but it is hard to think of any other beings which share a similar ontology. And of course, a simple enumeration of elements which fulfil a certain category, is a poor way of understanding the category itself. However, we may begin our inquiry by looking at the inverse of our question -- by asking: \enquote{What sort of being, is one who is \emph{not} a question of its being?} If a being is not a question of its being, then wouldn't it simply \emph{be} its being? Stripping away the rather unfortunate contortions of our language, a being that \emph{is} its being would simply be an identity:

\begin{equation}
    A = A
\end{equation}

\noindent
Identity is not just a simple unity of being, but the highest degree of \enquote{adequation} possible in ontology. Sartre is specific in stressing the absolute totality of an identity's self-coincidence, putting it as the logical limit unification:

\blockcquote[123]{Sartre}{%
    We are able to characterise the principle of identity as \enquote{synthetic,} not only because its scope is limited to a particular region but above all \emph{because it gathers within itself the infinity of density.} \enquote{A is A} means: A exists in the form of infinite compression, in an infinite density. Identity is the limiting concept of unification \ldots\ at its extreme limit, unity vanishes and passes over into identity.
}

\noindent
This \enquote{infinite density} of being comes from our original rejection of dualism within our ontology -- there can be nothing hidden behind being. \textcquote[123]{Sartre}{That adequation, which belongs to the in-itself, is expressed in this simple formula: being \emph{is} what it is. In the in-itself there is not a single particle of being with any distance from itself.} Thus, we are introduced to the first category of being, the kind which does not contain any question of itself -- the kind of being that simply \emph{is} itself, \emph{in} itself. This is the \emph{being-in-itself} of Sartre's metaphysics. 

\blockcquote[123]{Sartre}{%
    The [being] \emph{in-itself} is full of itself, and it is impossible to imagine a more complete plenitude, a more perfect adequation of any content to its container: there is not the slightest emptiness in being, not the slightest fissure through which nothingness might slip.
}

% TODO: make a TiKZ diagram showing Difference, Unity, and Identity as Venn Diagrams.

\noindent
What are some examples of this new category of being? The \emph{being-in-itself} is the being of material things, of simple existents. The chair, the table, the glass of apricot cocktails at the café -- the ontology of all these things is the being-in-itself. After all, an apricot cocktail \emph{is} an apricot cocktail -- it's being is infinite in its density, there can not be anything more to it or less. There is no \enquote{fissure through which nothingness might slip.}

Hence, if the infinite density of being is definitive of the being-in-itself, perhaps we can investigate the ontology of the being \enquote{who is a question of its own being} by looking at how it differs. The only example of such a being in this category is our being-of-consciousness, to which Sartre turns to in his examination: \textcquote[123]{Sartre}{Consciousness is characterised, on the contrary, by its decompression of being. Indeed, it is impossible to define it as self-coincident.} What does it mean for consciousness to not be self-coincident, to be a \textquote{decompression of being?} What is the ontology of the being-of-consciousness, in this light?

% TODO: In the closing wrap-up of this section, talk about how this is the baryon assymetry of our baryogenesis -- it is through this being-for-itself in which we see nothingness and contingency emerge.

% Maybe also spend a paragraph talking about temporality?

%  2. Being for-itself
%      1. We'll need to begin with an examination self-presence, as the 'Baryon asymmetry' of ontology 

\section{Being-for-the-Other}

%  3. Being for the Other
%      1. We'll have to examine how without the other, the world is merely 'psychological'


\chapter{The Contingency of Human Existence}

% Make sure to remember to talk about the following concept: the 'ontological-complexity' of a metaphysics. Where there is roughly a balance to be had between solipsism, and nihilism, but a complex array of options between the two which defines ontological complexity:
% 
% Solipsism - the ontology is too simple, all is reduced to one
% Finite ontology - the ontology of chess and checkers, a finite universe.
% Deterministic ontology -- the ontological complexity expands at a constant rate
% A Free ontology - the ontology expands non-linearly, but just right -- there is freedom
% Nihilism - the ontology is too general, there is no being at all

\chapter{The Teleology of Contingency}

% What is the relationship between existentialist ontology and morality? And philosophy? Are these things possible at all?

\chapter{Conclusion}
