\chapter{The Cosmogony of Sartrean Ontology}

% Explain briefly how the question of contingency comes from nothingness-- hence in order for us to explore this question properly, we need to first understand where nothingness comes from 

% Begin an explanation of the genesis of contingency from nothingness, using a physical analogy of Cosmogony.

% So what is the cosmogony of sartrean ontology? We're basically using the literary device of an extended metaphor with (physical) cosmogony, of an 'ontological Big-Bang' to set up a ordered exposition of Sartre's ontology.

% This extended metaphor will take us through the first three major parts of Sartre's Being and Nothingness:
%  1. The Problem of Nothingness -- where it comes from
%      1. First, we begin with existents -- and the phenomena which we perceive
%      2. We acknowledge the 'factual neccessity' of there being a consciousness to perceive phenomena
%      3. Next, we present the problem where phenomena must have its own being independent of the being of conscousness
%      4. We present the 'Baryogenesis' of ontology, where how the negative act of consciousness begets objectivity.
%  2. Being for-itself
%      1. We'll need to begin with an examination self-presence, as the 'Baryon asymmetry' of ontology 
%  3. Being for the Other
%      1. We'll have to examine how without the other, the world is merely 'psychological'
% We will save the last part for later