\chapter{Introduction}

% Why is this inquiry important in the first place?

% This paragraph essentially answers the question of "Why is metaphysics important?"

What is \emph{being?} \textquote{Why are there beings at all, instead of nothing?} That was the question, presented in the words of Martin Heidegger, as the \emph{fundamental question of metaphysics} \autocite[1]{Heidegger}. The question of being is both the broadest question, as well as the deepest -- those answer must account both for \textcquote[4]{Heidegger}{some elephant in a jungle in India, just as much as some chemical oxidation process on the planet Mars.}\footnote{First published in 1935, Martin Heidegger's astronomical quip predated the Viking lander (and any practical investigations into the being of Martian surface chemistry) by more than 41 years.} Our ability to give an adequate account of being is important, not just on the basis of some abstract, theoretical desire, but as a matter of practical utility too -- for to understand the \emph{being} of a human, is to know what is the \textcquote[11]{Aristotle}{characteristic activity} of a \emph{human} being. The broad and fundamental generality that Metaphysics holds in relation to the rest of Philosophy is akin to the relationship between Physics and Engineering -- for to understand the laws of the former would allow us to derive the facts of the latter. This makes ontology akin to the theoretical physics of Philosophy, with \emph{being} as the constitutive element of its science -- the \emph{elementary particle} of Metaphysics.

% The next paragraph essentially answers the question of "Why is Sartre's Metaphysics a good explanation, as opposed to Kant or anyone else?"

It is this question of being which first captured my interest, as a strong theoretical foundation in ontology can lead to further (and even unexpected) applications in more subsequent branches of philosophy. Jean-Paul Sartre's \emph{Being and Nothingness} is a monograph which presents a complete, and self-sufficient \emph{system} of ontology, that offers a stronger theoretical underpinning than prior systems we have studied. \emph{Being and Nothingness} descends from a phenomenological background which explicitly aims \textcquote[1]{Sartre}{to eliminate a number of troublesome dualisms from philosophy, and to replace them with the monism of the phenomenon.} This approach is entirely different, when compared with Immanuel Kant's transcendental metaphysics, whose \emph{Critique of Pure Reason}\footnote{The \emph{Critique of Pure Reason} was the subject of my \href{https://github.com/ShenZhouHong/kant-metaphysics}{2020-2021 Junior Year paper}.} presents and is dependent on a strong and inseparable dichotomy between the noumena and phenomena -- a \enquote*{troubling dualism,} in other words. Sartre rejects this dualism: we postulate that the being of an existent is entirely in the existent's appearances, and within a few short strokes lay out the opening propositions of an entirely novel \emph{phenomenological ontology.}

% Now we present the question of contingency, that comes from Sartre's ontology

What is the primary difference between Sartre's ontological system, and his phenomenological predecessors, such as Husserl or Heidegger? What is the chief, theoretical innovation, which distinguishes \emph{Being and Nothingness} from other antecedent theories of phenomenology? It is in the place of \emph{Nothingness}, which is central in Sartre's work. Nothingness is neither just a theme, nor a motif -- it is the \emph{fundamental force} of Sartrean metaphysics. It is the relationship between \emph{being} and \emph{nothingness}, like the interactions of an elementary particle in a physical field -- that yields the rich and vibrant account of human-reality which Sartre presents. Yet, out of all interactions between being and nothingness, it is the interaction between \emph{our} being, and nothingness, which both fascinates me, and troubles me.
For the same nothingness that yields phenomena and consciousness, freedom, and time -- also appears to exclude love, as a possibility within our metaphysics. \emph{Sartre's phenomenological ontology destroys our ordinary intuitions about Love, as a possible relation with the Other.} How can there be love, if the pursuit of the Other is ultimately self-defeating? What does it mean, to claim that our concrete relations with the Other, are ultimately a result of our struggle with our own \emph{contingent} being?
% For nothingness is not theme nor motif, not even \enquote*{just} a framework (no matter how essential) of Sartre's theoretical system. Nothingness is a necessary and \emph{irreducible} component of \emph{our} ontology, of the very being of \emph{our} self. It is like the vacuum which allows motion, enabling displacement in ontological space. This metaphor of motion is appropriate, for the relationship between nothingness and our own being is not a static one -- but rather, dynamic and troubled.

% TODO: Present the question on love, which is interesting and deserves treatment. As per suggestions from advisor:
% "Just anticipate a bit and say why it's troubling: 
% because it destroys our ordinary ethical intuitions about love.
% Love can no longer be what we thought it was."
%
% Might want to change the below paragraph as well, to fit the presentation of love as a question.

%Complex, turbulent, and mysterious -- this is the relationship of \emph{contingency} which I explore in this paper. 
Complex, turbulent, and troubled -- I begin the exploration of my question by examining the role of \emph{contingency} in Sartre's metaphysics.
Contingency is the uneasy relationship of our being-for-itself to it's being-in-itself. Unlike all other beings, we alone are not the \emph{foundation} of our being -- but rather, our being is defined by \emph{a question of its own being}. This contingency of the being-for-itself seems to be at the core of human-reality. It is from contingency which we derive all the vivid modalities of what it means to be human: the experiences of anguish, desire, love and hatred. How can this one characteristic of our ontology lead to such visceral experiences, which seem definitive of our very humanity? Likewise, how can contingency exclude the possibility of love? By looking at the question of contingency, we will be able to visit all the major landmarks of Sartre's ontological landscape, and in the process -- come to a better understanding of our own human nature. In this exploration of love, contingency, and humanity, I will take us on a journey that takes us to the heart of nothingness. The very nothingness that, \textcquote[57]{Sartre}{is neither before being nor after being; nor is it, in a general way, outside being; rather, it is right inside being, \emph{in its heart, like a worm.}}