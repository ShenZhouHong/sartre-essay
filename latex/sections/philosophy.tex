\chapter{Love, as a Possibility Only Through Philosophy}

In this penultimate section of our exploration in Sartrean Metaphysics, we revisit the possibility of love. In Sartre's \emph{Being and Nothingness}, we are presented with a demonstration on the impossibility of love as a project, an antinomy that we have recounted in \autoref{chap:love}. Yet, although Sartre has concluded his account of love, we cannot remain silent -- for his account is \emph{incomplete}, both on a theoretical basis, and as a matter of practical historiography. The impossibility of love as a relation with the Other is ontological, but only insofar as love is \emph{a project to found our contingency within the Other.} It is in the struggle with the contingency at the heart of our being, from which the First and Second Attitudes Towards the Other emerge. And it is because their birth stems from a struggle with contingency, that the aims of their projects will ultimately be unfulfillable. However, not all relationships with the Other are defined by the struggle with contingency, and likewise -- \emph{not all relationships of love are contingent.}

For there is one modality of our human-reality which \emph{is a love}, as it is a \emph{pursuit of a foundation for our being}, fulfilling the formal structure of the First and Second Attitudes. And yet, it is not a \emph{contingent} love, insofar contingency is the quality of possessing its own negation (i.e. self-defeating). Furthermore, this modality of love is not derivative, but rather is a fundamental component of our human-reality. This is the \emph{love of wisdom}, or \emph{Philosophy}.

Philosophy is the key to resolving the antinomy of love in Sartre's ontology, for it is both an exemplar, but also a means of \emph{reconciliation.} Philosophy is an exemplar of love, because it is a pursuit of the in-itself-for-itself, a divinity. This makes Philosophy a fulfilment of our original ontological act, in our desire to become the source of our own foundation (i.e. the Prime Mover). It is a reconciliation, because Philosophy is not a relation between a lone for-itself and an abstract transcendence. But rather, \emph{Philosophy is a concrete relation with the Other,} a practice between fellow for-itselfs, that necessitates interaction. This is why Philosophy as a practice is defined by its dialogues and dialectics -- why the Socratic method is a method of philosophical inquiry, or the seminar the foundation of the philosophical Academy. 

Hence, there is an ultimate reconciliation for the Sartrean antinomy of love, and this reconciliation lies within the hands of Philosophy. Philosophy is necessary for \enquote*{true} love, true in the sense of a love that is not self-defeating. A lover and the beloved can never find themselves in adequation, unless they are united by the pursuit of Philosophy as the intermediary term. Because only in such a way, do we pursue a love that is not an \emph{instrumental} love, where the beloved is placed in the antinomy of being the instrument of our foundation. But rather, through Philosophy -- we respect the fundamental ontology of the Other as a transcendence, \emph{as a connection between human-realities.} \textcquote[338]{Sartre}{Our relation is not a frontal opposition; rather it is an interdependence \emph{alongside.}} Therefore, we come to the conclusion that \emph{love is possible, within the ontology of Sartre's phenomenological metaphysics.} 

\noindent
Love, as a possibility -- only through the means of Philosophy.
