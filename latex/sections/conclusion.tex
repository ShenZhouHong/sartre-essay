\chapter{Conclusion}

Thank you for reading my thesis on the ontology of Sartrean metaphysics. In this exploration of Jean-Paul Sartre's \emph{Being and Nothingness}, we visited nearly every landmark of his phenomenological ontology. Beginning with the Cosmogony of Sartrean Ontology, we present the genesis of Sartre's philosophical system through a metaphor with physical cosmology -- starting with the foundation of phenomena, and ultimately deriving the being-for-itself, and the being-in-itself -- the elementary particles of Sartre's metaphysics.

Following that introduction, we begin our investigation of contingency -- a quality of our being-for-itself which stems from the nothingness that is central to our ontology. In this inquiry, we examine anguish as an immediate, personal conception of nothingness -- but only as a prelude, to the more visceral experience of existential terror, when we meet the Other. We unpack this existential terror, which stems from the particular contingency of the being-for-the-Other. We then follow up on the possibility of salvation, of finding a foundation for our being -- through the means of the Other.

It is through this understanding of the Other, that we are lead to the meat of our philosophical inquiry -- an examination of Love under Sartrean Ontology. We first review the antinomy of love, and see how all attempts at finding a foundation for our contingency through the Other fail, as they are ultimately self-defeating. It is this apparent antinomy of love which leads us to the key question of our thesis -- on whether love is possible at all. And through the examination of Philosophy as a love of wisdom, we come to the ultimate conclusion of our thesis.

Sartre's phenomenological ontology is fascinating and attractive, because it explains comprehensively, almost every aspect of human-reality. Through the humble beginnings of the phenomena, we derive a systematic theory that manages to persuasively account for elements of our human experience as varied as knowledge, faith, and sexuality. Sartre's \emph{Existentialism} is also a deeply affirming philosophy -- those steadfast belief in the integrity of our freedom offers a positive and humanist perspective to life. But yet, the apparent impossibility of love under Sartre's metaphysics colours its benefits, and makes the system as a whole all the more difficult to accept. It is this apparent discongruity in Sartre's account of love which lead me to this question -- among other reasons -- and in my thesis, I demonstrate that Philosophy is the only concrete relation with the Other which allows love to be possible.

However, throughout this entire journey, there was one last question, a puzzlement -- that I was not able to answer until now. How comes Sartre himself, did not present Philosophy as the relation in which love is possible? Why was he not aware of the same parallels, between the project of love, and the project of wisdom that allows one to reconcile both? I was puzzled by this apparent oversight, and indeed -- incredulous at the possibility of Sartre not being aware of this synthesis. It was only later did I realise, that Sartre did not overlook philosophy, as the precondition for love's possibility. In fact, he was aware of it all along. Sartre knew that love can only be accomplished through a relationship of Philosophy, and this was a truth that he \emph{practised}, in the very ontology of the life that he led.

For Jean-Paul Sartre embodied the necessary union of love and Philosophy, in his relationship with Simone de Beauvoir. As partners, equals, and fellow pursuers of truth -- their legendary friendship and companionship defies all definition. They did not seek in each other the world \emph{through} the being-of-the-Other, a world of dead-possibilities, of a transcended-transcendence. But rather, they saw each other as fellow lovers of freedom -- on the pursuit of apotheosis. And on that foundation of our original ontological act, Sartre and de Beauvoir faced the world instead, through all the modalities of experience that human-reality can offer. Sartre is an Existentialist, insofar he believed that his actions came from his own \emph{being}. And his ontology -- the now complete, totalising in-itself of his own historicity -- can testify that he knew this demonstration, and practised it with his very life.

This is why I choose Jean-Paul Sartre's \emph{Being and Nothingness}, as the source of my own explorations. For the question of being is the fundamental question of metaphysics, one that teaches us not only the structure of our human-reality as we experience it, but also provide valuable hints for our own ontological relations. I am a being, a for-itself. A first-class transcendence who is free on the basis of my ipseity. I am the author of my actions, and the source of my own non-being. The very non-being, through which, 

\noindent
I am free to \emph{be}.