\chapter{The Contingency of Human Existence}

% Be Goddamned passionate. This is the **KNIFE'S EDGE** of your thesis
% Talk about love.
% Getting out early what's at stake.
% The paradox: without contingency, we would be things. But WITH contingency, love and teleos becomes bleak.

We are contingent beings. This is the fundamental ontology of the human existence, the metaphysics of our humanity. A human being is no more than a being-for-itself, cloaked in the flimsy body of its own facticity. Our being is defined by our \emph{ontological act}, the constant fleeing of our for-itself away from the in-itself of facticity. Why are we contingent? What is the meaning of this contingency? What is at stake here, when our human-existence is defined by our own nothingness, which haunts us in every step we take? Is our contingency is like a curse, a hidden unravelling? Unlike the perfect adequation of the being-in-itself, we can \emph{never} hope to coincide with our being, to truly \emph{be} in-itself. We all contain the poisonous seed of negation, the question of our own being. This uneasiness manifests itself as the phenomena of \emph{anguish}, which Sartre dissects and presents to us, with all the cold certainty of a tumour at the hands of a pathologist. For what is anguish, other than a deep, visceral acknowledgement, of the contingency of our being?

% Talk about anguish as an introduction to the bad part of our contingency

Sartre is clear to distinguish anguish from the more ordinary emotion of fear. \textcquote[66]{Sartre}{Anguish is distinguished from fear by the fact that fear is fear of beings in the world, and anguish is anguish before myself.} It is this reflective nature, of an \enquote{anguish before myself,} that holds the key to the ontological meaning of this emotion.
\textcquote[66]{Sartre}{The preparation of artillery that precedes an attack may provoke fear in the soldier undergoing the bombardment, but he will begin to feel anguish when he tries to foresee the behaviour through which he will resist the bombardment, when he wonders if he will be able to \enquote{hold out.}} 
Will I hold out? Or will I crumple, under the stress of my situation?
That question, that uncertainty -- is a hesitation which points towards our being. For I am not steel nor iron, a material object.
The magnitude of my strength is not a rating, a measurement of kilopascals or newton-meters. If I was a being-in-itself, I would simply perform to the adequation of my being, with no questions involved. An iron bar has a tensile modulus, a compressive strength. It will always \emph{be} those properties, never more, but also never any less.

In contrast, as the being-for-itself, I always have the possibility of \emph{not} being, my being. The soldier undergoing bombardment might be trained to withstand fire, to have undergone extensive drills and exercises. Indeed, the soldier should have all the means of resisting the attack, as fortitude is characteristic of a soldier-being. But yet, at the moment of the bombardment, we realise that our resistance is just as equally \emph{possible} as our capitulation. There is no \emph{foundation} for our being, no matter how much we wish it to be. The soldier possesses every opportunity of abandoning his or her post, just as much as they possess the possibility of remaining steadfast upon their ground. \textcquote[69]{Sartre}{In other words, by constituting a specific course of action as \emph{possible}, and precisely because it is \emph{my} possible, I realise that \emph{nothing} can oblige me to take this action.} Nothing can oblige me to be who I am -- no argument can be made from determinism or \enquote{psychophysiological parallelism.} It is from this very nothingness, this negation of the self -- that we found the source of our own \emph{freedom.} \textcquote[66]{Sartre}{It is in anguish that man becomes conscious of his freedom, or alternatively, anguish is freedom's mode of being as consciousness of being; it is in anguish that freedom is, in its being, in question for itself.}

% Segue into talking about freedom, and how freedom arises from our contingency since we do not have a fixed being.

\section{Contingency as the Source of Freedom}

Hence, we gain an understanding of one of the roles that contingency serves in our metaphysics. Human beings are contingent, because it is through the contingency of our being that achieve a transcendence away from the in-itself of our facticity. We are free, precisely because we are free \emph{to not be}. \textcquote[578]{Sartre}{Man is free because he is not an in-itself, but a self-presence. A being that is what it is cannot be free. Freedom is precisely the nothingness that is \emph{been} at the heart of man, and which obliges human-reality to \emph{make itself}, rather than \emph{to be.}} Hence, this \enquote{unravelling} of our being appears to be an unexpected boon -- for the perfect, pure coincidence of being-in-itself lacks freedom. This contingency is the difference between our being, and the being of an automata -- of any artifice of intelligence that functions purely through deterministic means. For no matter how sophisticated an automaton, it will always act to the adequation of its being -- its motions and output the terminus of a long series of decision-trees. A being-in-itself, is, and always will -- be the sum of its parts.

% This section can be expanded, if I have time.

% We alone, are free to transcend our material existence. The question at the heart of our being, forever posits a negation of our-being, something that is not the sum of our parts. This is the source of a certain existential yearning -- which Sartre presents with the language of flight. But what are we fleeing towards, and what are we fleeing from? 

But why contingency? Why found our freedom in non-being? The Metaphysician must posit a mechanism to account for freedom, insofar that it exists. However, why did Sartre choose to found our freedom on \emph{nothingness?} On one hand, there is a certain grounding that can be achieved, if freedom is founded upon some principle that's as inherent as our ontology. For if we are free on the basis of our being, then nothing can remove such freedom from us, as long as we still \emph{are} in our being. \textcquote[579]{Sartre}{In this way freedom is not \emph{a} being: it is man's being, i.e. his nothingness of being.} This effectively grounds our freedom into the bedrock of our ontology, making it inseparable from our humanity. But this freedom comes at a cost, at a deep ontological insecurity. \textcquote[579]{Sartre}{Human-reality is entirely abandoned, without help of any kind, to the unbearable necessity of making itself be, right down to the last detail.} This abandonment is an ontological neurosis, a fundamental anxiety of our self. And there is no better manifestation of our existential anxiety -- than in our relationships with the Other. This is what we will examine next, in our quest to understand contingency, and its effects of the human condition. We will explore our relation with our being, when the for-itself is confronted -- by an \emph{Other} that is not itself.
% Talk about the bad part of freedom, and then transition into talking about concrete relations w/ the Other

\section{Contingency in Relations with the Other}

What does it mean to be confronted by the Other? What does it mean to meet a being in the world, that is both not a being-in-itself, but also not myself? In order to explore this relation, we must leave the bustling alterity of the modern world -- of soldiers and \enquote*{artillery preparations,} and enter the primitive existence of a simpler time. Imagine yourself, at the dawn of Sartre's baryogenesis. You are a for-itself, a first-class transcendence who is the source of your own nothingness. You live in a world of phenomena -- of bubbling brooks and shady oak groves, all of which act according to principles \enquote{not of your own will.} What sort of world is this? What manner of being is available, to the primordial for-itself?

It is an objective world, firstly. We have already achieved the baryosynthesis of the for-itself and the in-itself, and thus, there is no fear of regression into solipsism. It is also a free world, for you, as the for-itself, are free to seek projects of your own being within this world of phenomena. Through these projects, which are founded on the inalienable freedom of your human being, you encounter \emph{meaning} in the world -- as your being grasps its situation in the light of its aims and desires. The bubbling brook is a plenitude of being insofar as it is a being-in-itself, but it can mean fresh water, a respite from heat, or a treacherous obstacle, depending on whatever aims you project. There is still anguish, which may manifest during moments of particular danger or daring. Will I escape that bear? Am I brave enough to reach that nest? This is a contingent world. But it is also fundamentally a \emph{psychological} world. Even though beings-in-itself have a certain opacity, it is a harmless one -- for they possess no interior. There is no dualism between phenomena and being. They may have a being independent of your will, but their meaning is only a reflection of your projects, and thus, you alone are the source of all meaning, as certain as you are free.

%And through your projects, you encounter \emph{meaning} in the world -- as your being grasps its situation in the light of its aims and desires. 

% In talking about Other, talk about love, and the paradox of it

% Make sure to remember to talk about the following concept: the 'ontological-complexity' of a metaphysics. Where there is roughly a balance to be had between solipsism, and nihilism, but a complex array of options between the two which defines ontological complexity:

\section{Ontological Complexity and Metaphysics}
 
% Solipsism - the ontology is too simple, all is reduced to one
% Finite ontology - the ontology of chess and checkers, a finite universe.
% Deterministic ontology -- the ontological complexity expands at a constant rate
% A Free ontology - the ontology expands non-linearly, but just right -- there is freedom
% Nihilism - the ontology is too general, there is no being at all

% Talk about how we are different from the Greek Gods

% This is a complete restart, an ab-initio attempt at talking about the contingency in the face of the Other.

As human beings, we possess many modalities of interaction, which brings forth the contingency that is at the heart of our being. The phenomenon of anguish was only one of the ways in which we are confronted by our own non-being. Anguish is perhaps the most private experience of contingency, but it is neither the only experience, nor even the most profound. The most \emph{visceral} experience of our contingency comes from the \emph{Other}. It is an experience of endangerment, of an \emph{existential fear} -- that begins at the very instance that we are captured by the Other's gaze. 

What does it mean to be a for-itself, who is suddenly confronted by an Other? What is the Other, from the perspective of the self? A for-itself is a transcendence -- we are a \ldots\