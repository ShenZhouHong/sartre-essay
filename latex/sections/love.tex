\chapter{Love under Sartrean Ontology}

% What is the relationship between existentialist ontology and morality? And philosophy? Are these things possible at all?
% \section{Is Philosophy Possible at all?}

% \section{The Paradox of Love}

% \section{The Apotheosis of the Human-Condition}

We arrive now, at the heart of the Sartrean Ontology. In our struggle with the contingency of our being, we find the possibility of finding a foundation for our being, through the freedom of the Other. This project is \emph{love} -- or rather, the ontological foundation for love, the \emph{ideal} of love behind every lover. What is this project? How does it look like? Why is it so important, that Sartre himself devotes nearly a forth of his treatise, to the possibility of this pursuit? The project of love always begins with the Others look -- it is through this look that we experience the flight of our being into the Other's world, and with that -- the possibility of a foundation. \textcquote[481]{Sartre}{the Other \emph{looks} at me, and in doing so, he holds the secret of my being, he knows what \emph{I am}.} With this, we begin our attempt to retrieve this \enquote*{secret}. \textcquote[481]{Sartre}{I can try to retrieve this freedom and to take hold of it \ldots\ if I were able, in effect, to assimilate this freedom that founds my being-in-itself, I would be, in relation to myself, my own foundation.} We are not forced to pursue love in our relation to the Other -- if anything, it is only one of the two \emph{Attitudes towards the Other}. A rejection of this pursuit yields a corresponding attitude
% \footnote{Indifference, Hatred, and Sadism -- the Second Attitude Towards the Other.}
with its own modalities and ontology, that is worthy of exploration in its own right. However, it is the possibility of Love under Sartrean ontology that is the most personally fascinating -- for the conclusions that we derive here can yield important revelations about our personal lives. What is love, in the light of our contingency? Is the pursuit of love possible at all? We will begin our exploration of love, in the manner of a reductio -- as an examination of the antinomy of love under Sartrean metaphysics.

\section{The Antinomy of Love}

\textcquote[485]{Sartre}{My project to reclaim my being can be fulfilled only if I take hold of that freedom, and reduce it to being a freedom that submits to my freedom.}
This is the starting point of our project. What does it mean to reduce a freedom to a being which submits to our own freedom? Sartre takes care to note that this is not an act of subjugation -- \textcquote[486]{Sartre}{the person who wants to be loved does not desire to subjugate the being he loves.}
But rather, this is a special kind of appropriation, a desire to \enquote{possess a freedom \emph{as} freedom.} This possession of a freedom \emph{as} a freedom is essential to the project of love, for the very being-for-the-Other that we seek, only exists within the contingency of the Other's freedom.

But this is an intolerable contingency -- it does not suffice for the lover to be merely an object in the other's freedom. Such an object is just like any other -- one that can be lost at any moment, or transcended at the whim of the beloved. The lover cannot allow themselves to be surpassed -- but have their being founded in the beloved's freedom, as the very limit of said freedom. 

\blockcquote[488]{Sartre}{[This is what] the lover fundamentally demands of the loved one: he does not want \emph{to act} on the Other's freedom, but to exist \emph{a priori} as this freedom's objective limit, i.e., to be given in a single stroke alongside it, even as it arises, as the limit that the freedom must accept to be free.}

\noindent
This demand is an ontological demand on the behalf of the lover, that is absolute and totalising. \textcquote[489]{Sartre}{Thus, to be want to be loved is to want to place onself beyond the entire system of values posited by the Other, \emph{as the condition} of all valuation and as the objective foundation of all values.} Only in this way, can we find a certain foundation as the being-for-the-Other. Where instead of being threatened, of being endangered by the Other's freedom -- we are founded upon it, as the \emph{precondition} of it. This is the highest degree of support possible for our being, and through that, we achieve our goal.

\blockcquote[491]{Sartre}{Where, before we were loved, we were troubled by this unjustified and unjustifiable protuberance that was our existence, \ldots\ now we feel that this existence has been reclaimed and willed, right down to the last detail, by an absolute freedom that is conditioned by it at the same time -- and that, along with our own freedom, we are willing ourselves. \emph{That is the basis of love's joy, when it exists, to feel ourselves justified in existing}}

This is the summary of love's projected end, if it is possible. \textcquote[492]{Sartre}{If we were able to internalise the whole system, \emph{we would be the foundation of ourselves.}} How does this look like? What does it mean to be the foundation of ourselves? Does this question even need to be asked? We would become gods! We would become beings that are the foundations of our own consciousness, an \emph{in-itself-for-itself}, a perfect ontology. \textcquote[734]{Sartre}{An in-itself whose relation to its facticity would be like the for-itself's relation to its motivations.} We would lose the nothingness of our contingency, and have our being assured just as much as we are free. Even death cannot stop us in our apotheosis -- for \textcquote[706]{Sartre}{my being-for-the-Other is a real being and if, after my demise, it remains between the Other's hands \ldots\ it is in the form of a real-dimension of my being.} This is the salvation that we seek in the Other, in the struggle with our contingency.

\noindent
Unfortunately, this is an \emph{impossible goal}, a self-defeating antinomy.

\section{Is Love Possible at All?}

The impossibility of love under Sartre's demonstration is thus: the only way for love's project to succeed, is if the beloved becomes the lover in turn. And this is only possible, if the beloved themselves form the project of love -- a pursuit of their own foundation in their beloved. This symmetry is an annihilation of the very possibility of the project of love, for the demands that the lover places on their Other is a demand of subjectivity. I found my being as the \emph{object} of the Other, in all that terrifying contingency. But the moment I ask for the Other to become a freedom \emph{of my freedom}, it loses the very transcendence that made it free. \textcquote[497]{Sartre}{Love is a contradictory attempt to overcome the \emph{de facto} negation [i.e. my contingency] while at the same time maintaining the internal negation [i.e. to be contingent in the Other's freedom]. I demand the Other to love me, and do all that I can to fulfill my project, but if she loves me, she radically disappoints me even in her love: I required her to found my being as a favoured object by maintaining herself in her pure subjectivity before me; and the moment she loves me, she experiences me as a subject.} This is the reason why love is an impossible project, as a means of seeking a foundation for our being. We seek to found ourselves in a transcendence, but in our very attempt to capture it -- we render it into an immanence. 

Thus, we complete the terms of this reductio, under the postulates of Sartre's demonstration. This is the most troubling aspect of Sartre's ontological metaphysics. For given the terms, Sartre demonstrates love is an impossibility -- a self-defeating pursuit. He claims that love is both \enquote{an illusion and an infinite referral,} where \textcquote[499]{Sartre}{the more someone loves me, the more I lose my \emph{being.}} Is love truly impossible, under such a light? How can we preserve our ordinary ethical intuitions of love, in the face of these revelations? In order to explore this question, we must first begin with the topic that love is perhaps most intimately related to. That being the love of wisdom, or \emph{Philosophy.}

\section{Is Philosophy Possible at All?}

If love is impossible, does this mean philosophy is likewise impossible? Clearly, there is some necessity for metaphysics under Sartrean ontology -- if our existence is defined by our being, than the investigation of being is necessary as a means to understand our human-reality. Sartre's presentation of \emph{methods} of metaphysics demonstrates its continued validity -- the \emph{Existential Psychoanalysis} \autocite[723]{Sartre} is a tool for practical ontology. However, such a human-reality seems to debase Philosophical pursuits down to the simple level of the constitutive sciences. Philosophy is a love of wisdom -- and the pursuit of Philosophy bears the same hallmarks of a search for a \emph{foundation}, in the face of our contingency. Surely, if love is a self-defeating antinomy, than Philosophy likewise would become an impossible pursuit -- a means of seeking a foundation through transcendence, that yields immanence. Such a deduction, if true -- would be even more unsettling than the impossibility of love -- for any metaphysics which precludes Philosophy, would be a metaphysics not worth being in at all.

% Remember to talk about how love in the mundane sense is a matter of bad faith

However, there seems to be an essential difference, between the project of love in Philosophy, and the project of love for the Other. Both are the same insofar as they are an attempt at finding a foundation for our being, at absolving us of our contingency. It is an action that is characteristic of our ontology, as our ontological act is a lack that seeks to be fulfilled. But in the pursuit of a foundation as the being-for-the-Other, we are trying to found our being in the freedom of a tame transcendence, who must both be free, but also be limited in this freedom. This is an antinomy, but even as the \emph{object} of our ontological act, it is an impossibility. When the being of my for-itself transcends towards its desire of being a \textcquote[735]{Sartre}{foundation of its own being-in-itself purely by the means of its own being conscious of itself,} I cannot let an external being serve as a surrogate for this \enquote{being conscious of [my own] itself.} 
To attempt to do this is an act of \emph{bad-faith}, a willed self-delusion. This is ultimately a negation of the very self which we sought to give a foundation to. \textcquote[117]{Sartre}{Bad faith's most basic act is to flee from something that is impossible to flee from: to flee from what one is.} 

The pursuit of Philosophy is different. The object of our pursuit is wisdom -- but what is wisdom? It is different from the pursuit of knowledge, which is the aim of the constitutive sciences. After all, knowledge is a relationship of the being-for-itself to the world of the in-itself\footnote{Specifically,  according to Sartre: \textcquote[246]{Sartre}{The only kind of knowledge is intuitive, \ldots\ [and] intuition is the presence of consciousness to the thing [object of intuition].}} Instead, wisdom is the understanding of the being of things -- both as the being of the world in metaphysics, but also the subsequent beings of human-reality. To ask \enquote{What is virtue} is for Aristotle the question of \enquote{What is a man} -- and likewise, even more derivative inquiries concerning political philosophy fundamentally boil down to a question of the \emph{ontos} of humanity -- the \enquote{natural state of man.} Hence, Philosophy is a \emph{pursuit} -- but it is an \emph{ontological pursuit,} a pursuit of being.

% All Philosophy is fundamentally theological - it leads us towards God.

So what is this ultimate being which we pursue? It will not be any derivative or subsequent being -- but rather, Philosophy will always aim to the source of being. Thus, the object of Philosophy will have to be the first being, that being which is both its own foundation, and the source of all subsequent beings. This is, to use Sartre's language: \textcquote[735]{Sartre}{the ideal of a consciousness that could be the foundation of its own being-in-itself purely by the means of conscious of itself. To this ideal, we can give the name \emph{God.}} This is the \emph{First Philosophy} of Descartes and Aristotle -- it is how all Philosophy is fundamentally \emph{theological}, as it is an attempt to strive towards that being which founds itself -- God.

In this examination of the aim of Philosophy, we come to unpack some very important parallels between Philosophy and the \emph{ontological act} of our being. Just like Philosophy, the for-itself aims at being, and just like Philosophy, the being that we aim for, is that being which is the source of its own foundation. This is what Sartre means, when he makes the audacious claim of our teleology as an apotheosis -- when he claims that man's project is to be god. \textcquote[735]{Sartre}{To be a man is to aim to be God; or alternately, man is fundamentally the desire to be God.} The pursuit of Philosophy is the same as the pursuit of our for-itself, and in this way, it validates our being. Hence, just as Sartre states that \textcquote[735]{Sartre}{human-reality's fundamental project is [become] God,} we likewise assert that as a method of this project, \emph{Philosophy is a fundamental component of human-reality}

