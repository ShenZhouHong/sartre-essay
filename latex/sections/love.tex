\chapter{Love under Sartrean Ontology}

% What is the relationship between existentialist ontology and morality? And philosophy? Are these things possible at all?
% \section{Is Philosophy Possible at all?}

% \section{The Paradox of Love}

% \section{The Apotheosis of the Human-Condition}

We arrive now, at the heart of the Sartrean Ontology. In our struggle with the contingency of our being, we find the possibility of finding a foundation for our being, through the freedom of the Other. This project is \emph{love} -- or rather, the ontological foundation for love, the \emph{ideal} of love behind every lover. What is this project? How does it look like? Why is it so important, that Sartre himself devotes nearly a forth of his treatise, to the possibility of this pursuit? The project of love always begins with the Others look -- it is through this look that we experience the flight of our being into the Other's world, and with that -- the possibility of a foundation. \textcquote[481]{Sartre}{the Other \emph{looks} at me, and in doing so, he holds the secret of my being, he knows what \emph{I am}.} With this, we begin our attempt to retrieve this \enquote*{secret}. \textcquote[481]{Sartre}{I can try to retrieve this freedom and to take hold of it \ldots\ if I were able, in effect, to assimilate this freedom that founds my being-in-itself, I would be, in relation to myself, my own foundation.} We are not forced to pursue love in our relation to the Other -- if anything, it is only one of the two \emph{Attitudes towards the Other}. A rejection of this pursuit yields a corresponding attitude with its own modalities and ontology, that is worthy of exploration in its own right. However, it is the possibility of Love under Sartrean ontology that is the most personally fascinating -- for the conclusions that we derive here can yield important revelations about our personal lives. What is love, in the light of our contingency? Is the pursuit of love possible at all? We will begin our exploration of love, in the manner of a reductio -- as an examination of the antinomy of love under Sartrean metaphysics.

\section{The Antinomy of Love}

I say, that love is possible.\footnote{This presentation takes the form of a reductio, hence this \enquote{I say} is the presentation of an impossibility.} How does this look like? This is our First Attitude Towards the Other \autocite[482]{Sartre}.

% \textcquote[492]{Sartre}{If we were able to internalise the whole system, \emph{we would be the foundation of ourselves.}} How does this look like? What does it mean to be the foundation of ourselves? Does this question even need to be asked? We would become gods! We would become beings that are the foundations of our own consciousness, an \emph{in-itself-for-itself}, a perfect ontology. \textcquote[734]{Sartre}{An in-itself whose relation to its facticity would be like the for-itself's relation to its motivations.} We would lose the nothingness of our contingency, and have our being assured just as much as we are free. Even Death cannot stop us in our apotheosis -- for \textcquote[706]{Sartre}{my being-for-the-Other is a real being and if, after my demise, it remains between the Other's hands \ldots\ it is in the form of a real-dimension of my being.}

\section{Is Philosophy Possible at All?}

\section{Love as a Possibility through Philosophy}